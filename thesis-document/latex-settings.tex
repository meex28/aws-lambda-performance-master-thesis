%***************************************************************************************************

% Podstawowe ustawienia języka, według którego formatowany będzie dokument
\usepackage[polish]{babel}
% \usepackage[english]{babel}

% Pakiet babel dla polskiego języka powoduje konflikt z pakietem amssymb.
% Polecenie '\lll' definiują oba pakiety - porządana jest druga definicja.
\let\lll\undefined

% W przypadku wielojęzykowości ustawia główny język dokumentu
\selectlanguage{polish}

% Kodowanie dokumentu
\usepackage[utf8]{inputenc} 
\usepackage[T1]{fontenc} 

% Dowolny rozmiar czcionek, kodowanie znaków
\usepackage{lmodern}

% Polskie wcięcia akapitów
\usepackage{indentfirst}

% Polskie formatowanie typograficzne
\frenchspacing

% Zapewnia liczne usprawnienia wyświetlania i organizacji matematycznych formuł. 
\usepackage{amsmath}

% Wprowadza rozszerzony zestaw symboli m.in. \leadsto
\usepackage{amssymb}

% Dodatkowa, ,,kręcona'' czcionka matematyczna
\usepackage{mathrsfs}

% Dodatkowe wsparcie dla środowiska mathbb, które nie wspiera domyślnie cyfr (\mathbb{})
%\usepackage{bbold}

% Fixes/improves amsmath
\usepackage{mathtools}


% ***************************************************************************************************
% Kolory  
% ***************************************************************************************************

% Umożliwia kolorowanie poszczególnych komórek tabeli
\usepackage[table, svgnames]{xcolor}% http://ctan.org/pkg/

% Umożliwia łatwą zmianę koloru linii w tabeli
\usepackage{tabu}

% Umożliwia rozszerzoną kontrolę nad kolorami.
\usepackage{xcolor}

% Definicje kolorów
\definecolor{lgray}{HTML}{9F9F9F}
\definecolor{dgray}{HTML}{5F5F5F}

% ***************************************************************************************************
% Algorytmy 
% ***************************************************************************************************

% Udostępnia środowisko do konstruowania pseudokodów
\usepackage[ruled,vlined,linesnumbered,longend,algochapter]{algorithm2e}

% Zamiana nazwy środowiska z domyślnej "Algorithm X" na "Pseudokod X"
\newenvironment{algorithm-custom}[1][htb]{
	\renewcommand{\algorithmcfname}{Algorithm}
	\begin{algorithm}[#1]%
	}{
\end{algorithm}
}

% Zmiana rozmiaru komentarzy
\newcommand\algcomment[1]{
	\footnotesize{#1}
}

% Ustawienie zadanego stylu dla komentarzy
\SetCommentSty{algcomment}

% Wyśrodkowana tylda
\usepackage{textcomp}%
\newcommand{\textapprox}{\raisebox{0.5ex}{\texttildelow}}

% Listowanie kodów źródłowych
\usepackage{listings} 
\renewcommand{\lstlistingname}{Kod źródłowy} % Polska nazwa listingu

% ***************************************************************************************************
% Marginesy 
% ***************************************************************************************************

% Ustawienia rozmiarów stron i ich marginesów
\usepackage[headheight=18pt, top=25mm, bottom=25mm, left=25mm, right=25mm]{geometry}

% Usunięcie górnego marginesu dla środowisk
\makeatletter
\setlength\@fptop{0\p@}	
\makeatother

% ***************************************************************************************************
% Styl 
% ***************************************************************************************************

% Definiuje środowisko 'titlingpage', które zapewnia pełną kontrolę nad układem strony tytułowej.
\usepackage{titling}

% Umożliwia modyfikowanie stylu spisu treści
\usepackage[subfigure]{tocloft}	

\tocloftpagestyle{tableOfContentStyle}

% Definiowanie własnych stylów nagłówków i/lub stopek
\usepackage{fancyhdr}

% Domyślny styl dla pracy 
\fancypagestyle{custom}{
	\fancyhf{}									% wyczyść stopki i nagłówki
	\fancyhead[RO]{								% Prawy, nieparzysty nagłówek
		\hrulefill \hspace{16pt} \large Rozdział \thechapter
		\put(-471.5,5.5){%
			\makebox(0,0)[l]{%
				\small Politechnika Wrocławska
			}
		}
	}
	\fancyhead[LE]{								% Lewy, parzysty nagłówek
		\large Rozdział \thechapter \hspace{16pt} \hrulefill 
		\put(-267,5.5){%
			\makebox(0,0)[l]{%
				\small Wydział Informatyki i Telekomunikacji
			}
		}
	}
	\fancyfoot[LE,RO]{							% Stopki
		\thepage
	}
	\renewcommand{\headrulewidth}{0pt}			% Grubość linii w nagłówku
	\renewcommand{\footrulewidth}{0.2pt}		% Grubość linii w stopce
}

% Domyślny styl dla List of Figures
\fancypagestyle{ListofFiguresStyle}{
	\fancyhf{}									% wyczyść stopki i nagłówki
	\fancyhead[RO]{								% Prawy, nieparzysty nagłówek
		\hrulefill \hspace{16pt} \large List of Figures
		\put(-471.5,5.5){%
			\makebox(0,0)[l]{%
				\small Politechnika Wrocławska
			}
		}
	}
	\fancyhead[LE]{								% Lewy, parzysty nagłówek
		\large List of Figures \hspace{16pt} \hrulefill 
		\put(-267,5.5){%
			\makebox(0,0)[l]{%
				\small Wydział Informatyki i Telekomunikacji
			}
		}
	}
	\fancyfoot[LE,RO]{							% Stopki
		\thepage
	}
	\renewcommand{\headrulewidth}{0pt}			% Grubość linii w nagłówku
	\renewcommand{\footrulewidth}{0.2pt}		% Grubość linii w stopce
}

% Domyślny styl dla List of Tables
\fancypagestyle{ListofTablesStyle}{
	\fancyhf{}									% wyczyść stopki i nagłówki
	\fancyhead[RO]{								% Prawy, nieparzysty nagłówek
		\hrulefill \hspace{16pt} \large List of Tables
		\put(-471.5,5.5){%
			\makebox(0,0)[l]{%
				\small Politechnika Wrocławska
			}
		}
	}
	\fancyhead[LE]{								% Lewy, parzysty nagłówek
		\large List of Tables \hspace{16pt} \hrulefill 
		\put(-267,5.5){%
			\makebox(0,0)[l]{%
				\small Wydział Informatyki i Telekomunikacji
			}
		}
	}
	\fancyfoot[LE,RO]{							% Stopki
		\thepage
	}
	\renewcommand{\headrulewidth}{0pt}			% Grubość linii w nagłówku
	\renewcommand{\footrulewidth}{0.2pt}		% Grubość linii w stopce
}


% Domyślny styl dla bibliografii
\fancypagestyle{bibliographyStyle}{
	\fancyhf{}									% wyczyść stopki i nagłówki
	\fancyhead[RO]{								% Prawy, nieparzysty nagłówek
		\hrulefill \hspace{16pt} \large Bibliografia
		\put(-471.5,5.5){%
			\makebox(0,0)[l]{%
				\small Politechnika Wrocławska
			}
		}
	}
	\fancyhead[LE]{								% Lewy, parzysty nagłówek
		\large Bibliografia \hspace{16pt} \hrulefill 
		\put(-267,5.5){%
			\makebox(0,0)[l]{%
				\small Wydział Informatyki i Telekomunikacji
			}
		}
	}
	\fancyfoot[LE,RO]{							% Stopki
		\thepage
	}
	\renewcommand{\headrulewidth}{0pt}			% Grubość linii w nagłówku
	\renewcommand{\footrulewidth}{0.2pt}		% Grubość linii w stopce
}

% Domyślny styl dla dodatków
\fancypagestyle{appendixStyle}{
	\fancyhf{}									% wyczyść stopki i nagłówki
	\fancyhead[RO]{								% Prawy, nieparzysty nagłówek
		\hrulefill \hspace{16pt} \large Appendix \thechapter
		\put(-472.1, 12.1){%
			\makebox(0,0)[l]{%
				\includegraphics[width=0.05\textwidth]{resources/pwr-logo}
			}
		}
		\put(-443,5.5){%
			\makebox(0,0)[l]{%
				\small Politechnika Wrocławska
			}
		}
	}
	\fancyhead[LE]{								% Lewy, parzysty nagłówek
		\large Dodatek \thechapter \hspace{16pt} \hrulefill 
		\put(-22, 12.1){%
			\makebox(0,0)[l]{%
				\includegraphics[width=0.05\textwidth]{wiz-logo}
			}
		}
		\put(-220,5.5){%
			\makebox(0,0)[l]{%
				\small Wydział Informatyki i Telekomunikacji
			}
		}
	}
	\fancyfoot[LE,RO]{							% Stopki
		\thepage
	}
	\renewcommand{\headrulewidth}{0pt}			% Grubość linii w nagłówku
	\renewcommand{\footrulewidth}{0.2pt}		% Grubość linii w stopce
}


% Osobny styl dla stron zaczynających rozdział/spis treści itd. (domyślnie formatowane jako "plain")
\fancypagestyle{chapterBeginStyle}{
	\fancyhf{}%
	\fancyfoot[LE,RO]{
		\thepage
	}
	\renewcommand{\headrulewidth}{0pt}
	\renewcommand{\footrulewidth}{0.2pt}
}

% Styl dla pozostałych stron spisu treści
\fancypagestyle{tableOfContentStyle}{
	\fancyhf{}%
	\fancyfoot[LE,RO]{
		\thepage
	}
	\renewcommand{\headrulewidth}{0pt}
	\renewcommand{\footrulewidth}{0.2pt}
}

% Formatowanie tytułów rozdziałów i/lub sekcji
\usepackage{titlesec}


% Formatowanie tytułów rozdziałów
\titleformat{\chapter}[hang]
{\vspace{-10ex}\normalfont\Huge\bfseries}											
{\thechapter.}
{1ex}
{} 
%[\vspace{2ex}]

% Formatowanie tytułów sekcji
\titleformat{\section}[hang]
{\normalfont\Large\bfseries}											
{\thesection.}
{1ex}
{} 

\titleformat{\subsection}[hang]
{\normalfont\large\bfseries}											
{\thesubsection.}
{1ex}
{} 
% formatowanie elementów przed modyfikowanym tytułem


% ***************************************************************************************************
% Linki
% ***************************************************************************************************

% Umożliwia wstawianie hiperłączy do dokumentu
\usepackage{hyperref}							% Aktywuje linki

\hypersetup{
	colorlinks	=	true,					% Koloruje tekst zamiast tworzyć ramki.
	linkcolor	=	blue,					% Kolory: referencji,
    citecolor	=	blue,					% cytowań,
	urlcolor	=	blue					% hiperlinków.
}

% Do stworzenia hiperłączy zostanie użyta ta sama (same) czcionka co dla reszty dokumentu
\urlstyle{same}


% ***************************************************************************************************
% Linki
% ***************************************************************************************************

% Umożliwia zdefiniowanie własnego stylu wyliczeniowego
\usepackage{enumitem}

% Nowa lista numerowana z trzema poziomami
\newlist{myitemize}{itemize}{3}

% Definicja wyglądu znacznika pierwszego poziomu
\setlist[myitemize,1]{
	label		=	\textbullet,
	leftmargin	=	4mm}

% Definicja wyglądu znacznika drugiego poziomu
\setlist[myitemize,2]{
	label		=	$\diamond$,
	leftmargin	=	8mm}

% Definicja wyglądu znacznika trzeciego poziomu
\setlist[myitemize,3]{
	label		=	$\diamond$,
	leftmargin	=	12mm
}

% ***************************************************************************************************
% Inne pakiety
% ***************************************************************************************************

% Dołączanie rysunków
\usepackage{graphicx}

% Figury i przypisy
\usepackage{caption}
%\usepackage{subcaption}

% Umożliwia tworzenie przypisów wewnątrz środowisk
\usepackage{footnote}

% Umożliwia tworzenie struktur katalogów
\usepackage{dirtree}

% Rozciąganie komórek tabeli na wiele wierszy
\usepackage{multirow}

% Precyzyjne obliczenia szerokości/wysokości dowolnego fragmentu wygenerowanego przez LaTeX
\usepackage{calc}

% ***************************************************************************************************
% Matematyczne skróty
% ***************************************************************************************************

% Skrócony symbol liczb rzeczywistych
\newcommand{\RR}{\mathbb{R}}

% Skrócony symbol liczb naturalnych
\newcommand{\NN}{\mathbb{N}}

% Skrócony symbol liczb wymiernych
\newcommand{\QQ}{\mathbb{Q}}

% Skrócony symbol liczb całkowitych
\newcommand{\ZZ}{\mathbb{Z}}

% Skrócony symbol logicznej implikacji
\newcommand{\IMP}{\rightarrow}

% Skrócony symbol  logicznej równoważności
\newcommand{\IFF}{\leftrightarrow}

% ***************************************************************************************************
% Środowiska
% ***************************************************************************************************

% Środowisko do twierdzeń
\newtheorem{theorem}{Twierdzenie}[chapter]

% Środowisko do lematów
\newtheorem{lemma}{Lemat}[chapter]

% Środowisko do przykładów
\newtheorem{example}{Przykład}[chapter]

% Środowisko do wniosków
\newtheorem{corollary}{Wniosek}[chapter]

% Środowisko do definicji
\newtheorem{definition}{Definicja}[chapter]

% Środowisko do dowodów
\newenvironment{proof}{
	\par\noindent \textbf{Dowód.}
}{
\begin{flushright}
	\vspace*{-6mm}\mbox{$\blacklozenge$}
\end{flushright}
}

% Środowisko do uwag
\newenvironment{remark}{
	\bigskip \par\noindent \small \textbf{Uwaga.}
}{
\begin{small}
	\vspace*{4mm}
\end{small}
}

% dodatkowe pomagające, oczywiście nie wszystskie są wymagane
\usepackage{psfrag}
\usepackage{amsfonts}
\usepackage{supertabular}
\usepackage{array}
\usepackage{tabularx}
\usepackage{hhline}
\usepackage{minted}
\usepackage{url}
\usepackage{microtype}
\usepackage{booktabs} % for professional tables
\usepackage{makecell}
\usepackage{rotating}
\usepackage{multicol}
\usepackage{cuted}
\usepackage{colortbl}
\usepackage{adjustbox}
\usepackage{color,soul}
\usepackage{subfigure}
\usepackage{pdfpages}

\let\origdoublepage\cleardoublepage
\newcommand{\clearemptydoublepage}{\clearpage{\pagestyle{empty}\origdoublepage}}
\let\cleardoublepage\clearemptydoublepage

\usepackage{pifont}
\newcommand{\cmark}{\ding{51}}
\newcommand{\xmark}{\ding{55}}
\newcommand{\bftab}{\fontseries{b}\selectfont}

\newcolumntype{P}[1]{>{\raggedright\arraybackslash\noindent}p{#1}}


\newcolumntype{R}[1]{>{\raggedleft\arraybackslash}p{#1}}
\newcolumntype{L}[1]{>{\raggedright\arraybackslash}p{#1}}
\newcolumntype{C}[1]{>{\centering\arraybackslash}m{#1}}

\usepackage{caption}
\usepackage[nocompress]{cite}
\usepackage{url}
\usepackage{color,soul}
\usepackage{svg}
\usepackage{tabto}
\usepackage{wrapfig}

% formatowanie pierwszych stron rozdziałów - pomagające
\newcommand{\resetformatting}{ 
\fancypagestyle{plain}{
    	\fancyhf{}%
    	\fancyfoot[LE,RO]{
    		\thepage
    	}
    	\renewcommand{\headrulewidth}{0pt}
    	\renewcommand{\footrulewidth}{0.2pt}
    } }
\newcommand{\doublepage}{ 
\newpage
\thispagestyle{empty}
\cleardoublepage}

\newcommand\Chapter[1]{ 
\chapter{#1}
\thispagestyle{chapterBeginStyle}}