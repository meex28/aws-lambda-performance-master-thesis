\chapter{Wnioski i dyskusja}\label{chapter:wnioski_dyskusja}

% Jakiś wstęp

\section{Odpowiedzi na pytania badawcze}\label{chapter:odpowiedzi_na_pytania_badawcze}

W ramach podrozdziału udzielono odpowiedzi na pytania badawcze sformułowane w pracy. 
Odpowiedzi bazowały na wynikach opisanych w Rozdziale \ref{chapter:wyniki} oraz przypadku trzeciego pytania badawczego także na cechach metod opisanych w Rozdziale \ref{chapter:wybrane_metody_optimalizacji}.

\subsection*{Odpowiedź na pierwsze pytanie badawcze}

Pierwsze pytanie badawcze postawione w pracy brzmi: ,,Które metody optymalizacji pozwalają na najlepszą poprawę ogólnej wydajności funkcji AWS Lambda w ekosystemie Java?''.
W celu odpowiedzi w pracy zawarto cztery różne kryteria: czas działania funkcji podczas ciepłych startów, czas działania funkcji podczas zimnych startów, współczynnik wydajności funkcji oraz koszt działania funkcji.
Badania pod ich kątem wykazały, że efektywność poszczególnych metod jest bardzo zależna od konkretnego przypadku użycia (np. często lub rzadko używane funkcje) oraz wielkości pamięci.

W przypadku funkcji o wysokiej aktywności (czyli z dominacją ciepłych startów), żadna z badanych metod nie wykazała poprawy wydajności dla każdej z badanych wielkości pamięci.
Dla czasu ciepłych startów widoczny jest negatywny wpływ usługi SnapStart (zarówno w przypadku Javy, jak i Kotlina) oraz zastosowania języka interpretowanego (w przypadku użycia Kotlin/JS).
Metody bazujące na kompilacji kodu do natywnych plików binarnych są najbardziej obiecujące.
Są to platformy GraalVM oraz Kotlin/Native, których efektywność możemy rozpatrzyć pod kątem małych i dużych rozmiarów pamięci.
Dla małych rozmiarów pamięci (128 MB oraz 256 MB) Kotlin/Native zapewnił poprawę względem klasycznej funkcji Java JVM, o odpowiednio 49\% i 35\%.
GraalVM pozwolił na większą poprawę dla pamięci 128 MB (o 60\%), jednak w przypadku pamięci 256 MB cechował się wydłużeniem czasu odpowiedzi.
Dla pamięci 512 MB obie metody cechowały się dłuższym czasem procesowania niż klasyczna funkcja Java JVM, jednak GraalVM był znacznie szybszy (pogorszenie o 12\%, w porównaniu z 231\% dla Kotlin/Native).
W przypadku większych pamięci, czyli 1024 MB i 2048 MB, GraalVM skrócił czas procesowania o odpowiednio 40\% i 23\%.

Zatem dla funkcji AWS Lambda o wysokiej aktywności i wysokim ponownym użyciu rozgrzanych już instancji najbardziej uniwersalnym wyborem będzie użycie GraalVM.
Jednak w przypadku chęci użycia mniejszej pamięci, Kotlin/Native będzie sposobem bardziej stabilnym (ze względu na brak znacznego pogorszenia wydajności dla pamięci 256 MB).
Dodatkowo, metoda ta wykazuje bardzo wysoką poprawę efektywności kosztowej.
Dla rozmiarów pamięci 128 MB i 256 MB Kotlin/Native pozwolił na uzyskanie najniższych kosztów, zatem może to być istotny czynnik w wyborze metod optymalizacji przez zespoły programistyczne.

Analizując przypadek funkcji mniej aktywnych (czyli z większym udziałem zimnych startów), metody wykazywały inne właściwości niż dla ciepłych startów.
Poprzez ocenę współczynnika wydajności funkcji (WWF) można stwierdzić, że większość metod pozwoliły na pewną poprawę wydajności.
Jedynie dla mniejszych pamięci (128 MB, 256 MB) metody jak SnapStart i Kotlin prowadziły do pogorszenia wydajności (oprócz użycia Javy z aktywnym SnapStart i pamięci 256 MB).
Dla pamięci wielkości 128 MB, 256 MB ponownie najbardziej skuteczne okazały się metody oparte o kompilację do natywnych plików binarnych.
Szczególną uwagę przykuwa jednak użycie Kotlin/Native, który zapewnił znacznie większy wzrost wydajności niż GraalVM.
Wykazał się on najbardziej efektywną metodą dla funkcji o pamięci 128 MB, a także dla rozmiaru 256 MB, gdzie wzrost wydajności był kolosalny.
Kotlin/Native osiągnał trzykrotnie większą wartość współczynnika niż GraalVM, który był drugą najbardziej skuteczną metodą.
Czas uruchomienia funkcji dla pamięci 128 MB i 256 MB był także najniższy dla Kotlin/Native.
Zostanie to omówione szczegółowe w ramach odpowiedzi na drugie pytanie badawcze.
Dodatkowo, w ramach tych wartości pamięci możliwe było uzyskanie najniższych kosztów działania funkcji wśród wszystkich badanych konfiguracji.
Zostało to otrzymane dzięki użyciu kompilacji Kotlina do natywnych plików binarnych.
Wszystkie te aspekty wskazują na bardzo wysoką skuteczność tej metody w zakresie niewielkich rozmiarów pamięci (128 MB, 256 MB).

Dla aktywnych funkcji o rozmiarze pamięci od 512 MB do 2048 MB, trzy z badanych metod wykazały istotną skuteczność.
Były to Kotlin/JS, Kotlin/Native oraz GraalVM.
Dla pamięci 512 MB, metody wykazały zbliżoną skuteczność pod względem współczynnika wydajności funkcji (przy czym GraalVM był najbardziej wydajny).
Wraz z wzrostem pamięci rosła wydajność wszystkich trzech metod, jednak istotny jest tutaj wzrost wydajności dla funkcji używających Kotlin/Native.
Dla rozmiaru pamięci 1024 MB Kotlin/Native uzyskał najwyższą wartość współczynnika, o około 10\% wyższą niż kolejny GraalVM.
Jednak dla rozmiaru 2048 MB, różnica ta zdecydowanie zwiększyła się, gdzie Kotlin/Native osiągnął wynik o około 74\% wyższy od GraalVM.
Wartym uwagi jest także bardzo wysoka wydajność Kotlin/Native w konfiguracji pamięci 256 MB, gdzie było to połączenie o drugiej najwyższej wartości współczynnika.
Dodatkowo, Kotlin/Native cechował się wysoką efektywnością kosztową, gdzie pozwolił na uzyskanie niższego kosztu działania niż GraalVM we wszystkich z badanych rozmiarów pamięci.

Podsumowując, niemożliwe jest wybranie jednej najbardziej efektywnej metody optymalizacji wydajności dla wszystkich przypadków użycia funkcji AWS Lambda w ekosystemie Java.
Dla funkcji o bardzo wysokiej aktywności, GraalVM byłby najbardziej uniwersalną techniką, która jednak nie jest skuteczna dla każdej konfiguracji pamięci.
Podczas użycia funkcji o mniejszej aktywności i wyższym znaczeniu zimnych startów, bardzo sprawną techniką jest Kotlin/Native, który cechuje się także dobrą efektywnością kosztową, w porównaniu z alternatywnym GraalVM.
Wartym uwagi jest jednak przypadek funkcji o niewielkim rozmiarze pamięci (128 MB, 256 MB) i używających Kotlin/Native.
W ramach tych funkcji metoda ta pozwala na poprawę wydajności w ramach każdego z badanych aspektów.
Redukuje ona czas ciepłych startów, osiąga najniższy czas działania dla zimnych startów oraz najwyższy współczynnik wydajności funkcji w ramach tych rozmiarów pamięci.
Wszystko to jest możliwe przy jednocześnie najniższych z otrzymanych w badaniu kosztów działania.
Z tego względu Kotlin/Native powinien być rozważany w przypadku decyzji o wyborze metody optymalizacji wydajności.

\subsection*{Odpowiedź na drugie pytanie badawcze}

Drugim pytaniem badawczym sformułowane w pracy jest: ,,W jakim stopniu wybrane metody optymalizacji redukują czas zimnego startu funkcji Java w AWS Lambda?''.

\subsection*{Odpowiedź na trzecie pytanie badawcze}

Trzecim pytaniem badawczym, które zostało uwzględnione w pracy, jest: ,,Jakie kompromisy w procesie rozwoju oprogramowania wiążą się implementacją poszczególnych metod optymalizacji wydajności funkcji Java w AWS Lambda?''.