\chapter*{Zakończenie}\label{chapter:zakonczenie}
\addcontentsline{toc}{chapter}{Zakończenie}  

Celem pracy było zaproponowanie nowych metod poprawy wydajności funkcji AWS Lambda w ekosystemie Java oraz ich analiza.
Analiza została dokonana poprzez badanie wydajności funkcji oraz ich wpływu na inne czynniki, które mogą oddziaływać na proces rozwoju oprogramowania opartego o AWS Lambda.
W pracy zawarto dwie metody zidentyfikowane w ramach przeglądu, czyli SnapStart oraz GraalVM.
Kluczowym elementem pracy są jednak metody oparte o język Kotlin, które zostały zaproponowane jako nowe sposoby optymalizacji wydajności.
Nowo zaproponowane techniki to Kotlin JVM (także z aktywną funkcją SnapStart), Kotlin/JS oraz Kotlin/Native.
Wszystkie z poruszonych metod zostały zcharakteryzowane w ramach pracy, uwzględniając sposób ich działania, zalety oraz wady, co było ważnym elementem oceny ich wpływu na wytwarzanie oprogramowania.

W celu dokonania rzetelnej analizy skuteczności poszczególnych metod przygotowano eksperyment badający wydajność poszczególnych metod.
Dla każdej z omawianych metod przygotowano i opisano implementację funkcji AWS Lambda.
Następnie, funkcje te zostały wdrożone w różnych konfiguracjach pamięci, w ramach środowiska przedstawionego w opracowaniu.
Z jego użyciem wykonano pomiar wydajności w ramach metryk: czas działania funkcji podczas ciepłego i zimnego startu, zaproponowany w ramach pracy współczynnik wydajności funkcji oraz koszt działania.
Metryki te pozwoliły odpowiedzieć na dwa pierwsze pytania badawcze sformułowane w pracy, które poruszały poprawę ogólnej wydajności funkcji oraz redukcji czasu działania funkcji podczas zimnego startu.
W ramach pracy odpowiedziano także na trzecie pytanie badawcze, które analizowało oddziaływanie metod na proces rozwoju oprogramowania.
W tym celu przeanalizowano proces budowy artefaktów z użyciem poszczególnych metod, potencjalne ograniczenia w korzystaniu z AWS SDK, koszt działania funkcji oraz inne cechy, które zawarto w charakterystykach poszczególnych metod.

\subsection*{Wnioski}

W ramach pracy dokonano przeglądu literatury, który pozwolił na poznanie aktualnego stanu wiedzy na temat poprawy wydajności AWS Lambda w ekosystemie Java.
Na jego bazie włączono do pracy dwie istniejące już metody, które nie były jednak poruszone w wielu pracach, a cechowały się wysoką skutecznością.
Zidentyfikowano także obszar, który nie został jeszcze zbadany, czyli użycia języka Kotlin wraz z jego rozbudowanym zestawem narzędzi.
Dodatkowo, w przeglądzie literatury ustalono czynniki, które wpływają na wydajność funkcji, m.in. zimne starty i wielkość pamięci, co wykorzystano w badaniu.

Poprzez analizę wydajności funkcji pod względem różnych metryk oraz rozmiarów pamięci stwierdzono znaczne zróżnicowanie w skuteczności metod w zależności od przypadku użycia.
Dla funkcji aktywnych, czyli z większym udziałem ciepłych startów, zauważono, że metody oparte o kompilację do plików natywnych są najskuteczniejsze.
Stwierdzono, że GraalVM jest najbardziej uniwersalną metodą dla tego przypadku funkcji, jednak Kotlin/Native jest bardziej wyspecjalizowany dla funkcji o małym rozmiarze pamięci (128 MB, 256 MB).
Dla funkcji o mniejszej aktywności, czyli z znaczym udziałem zimnych startów, metody GraalVM, Kotlin/JS i Kotlin/Native wykazały się wysoką efektywnością.
Aby odwzorować wydajność funkcji zaproponowano współczynnik wydajności funkcji (WWF), który agregował średnie czasy wykonania podczas zarówno ciepłych, jak i zimnych startów.
Poprzez jego analizę stwierdzono, że Kotlin/Native jest najskuteczniejszą metodą dla większości rozmiarów pamięci.
Dodatkowo, zauważono, że Kotlin/Native cechował się najniższym czasem zimnych startów dla wszystkich badanych pamięci.
Co istotne, metoda ta pozwoliła na osiągnięcie najniższych możliwych kosztów wśród badanych metod.
W jej przypadku były one najniższe dla rozmiarów pamięci 128 MB i 256 MB.

Ważnym aspektem pracy jest także analiza użycia badanych metod na proces rozwoju oprogramowania.
Zidentyfikowano kompromisy, których muszą być świadomi programiści decydujący się na wdrożenie wybranych metod poprawy wydajności.
W pracy opisano potencjalne ograniczenia w korzystaniu z funkcji języka (jak w przypadku SnapStart, GraalVM, Kotlin/Native), czy negatywny wpływ na koszt działania.
Ważny może być także wydłużony czas budowy artefaktu dla metod GraalVM i Kotlin/Native, a także znacznie większy rozmiar plików wdrożeniowych w przypadku pierwszej metody.
Istotna może być jednak możliwość integracji z zewnętrznymi serwisami, która jest utrudniona w przypadku korzystania z metod Kotlin/JS i Kotlin/Native.

\subsection*{Zastosowanie pracy i perspektywy rozwoju}

Wyniki badań i analiz przeprowadzonych w ramach pracy mogą być znacznym wsparciem dla zespołów programistycznych tworzących aplikacje oparte o AWS Lambda i ekosystem Java.
Po pierwsze, świadomość o istnieniu analizowanych metod może być zachęceniem do użycia podczas tworzenia funkcji AWS Lambda języków Kotlin lub Java, zamiast JavaScript lub Python.
Po drugie, dostarczone wnioski pomogą w wyborze odpowiedniej metody w zależności od przypadku danego zespołu.
Poprzez poznanie pewnych kompromisów wynikających z tej decyzji, będzie ona także bardziej świadoma.
Zatem znaczącym zastosowaniem pracy jest bezpośrednie wsparcie zespołów programistycznych.

Istotnym elementem usług bezserwerowych jest koszt działania, który jest bezpośrednio powiązany z wydajnością.
Wyniki pracy mogą przyczynić się do ich redukcji, poprzez użycie odpowiednich metod.
Dostarczone wyniki w kontekście skuteczności poszczególnych sposobów w zależności od pamięci skutecznie wesprą programistów w konfiguracji jej rozmiaru w funkcjach AWS Lambda.

Opisane w pracy sposób przeprowadzenia eksperymentu oraz środowisko badawcze mogą być ponownie wykorzystane.
Może to posłużyć jako baza do dalszych badań naukowych ale także jako metoda ewaluacji wydajności różnych funkcji, które są wykorzystywane biznesowo przez programistów.
Odwzorowane na bazie charakterystyki środowisko badawcze pozwoli na stworzenie odpowiednich warunków do uruchomienia różnych typów funkcji i oceny ich wydajności.

Dalszymi kierunkami rozwoju dla tematu pracy może być bardziej dogłębna analiza części metod oraz przeprowadzenie badań w bardziej szczegółowych przypadkach użycia.
Interesującym obszarem są dalsze badania języka Kotlin w kontekście AWS Lambda, poprzez na przykład analizę narzędzi (np. bibliotek), który ten język oferuje.
Wynikiem skłaniającym do dalszych eksperymentów jest także bardzo niska skuteczność użycia usługi SnapStart w połączeniu z Kotlinem.
Jest to efekt inny niż zastosowanie z Javą, zatem poznanie przyczyn tego zjawiska może być elementem kolejnego badania.
Praca może być także rozwijana w zastosowań funkcji AWS Lambda, innych niż te poruszone w pracy.
Metody mogą zostać ocenione w innych scenariuszach, jak integracja z zewnętrznymi usługami, czy operacje wymagające dużej mocy obliczeniowej.
Praca skłania do dalszych badań i eksperymentów nad użyciem Kotlina w AWS Lambda, co może przyczynić się do większej popularyzacji tego języka.
