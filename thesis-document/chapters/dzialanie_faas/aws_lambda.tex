\section{AWS Lambda}\label{chapter:aws_lambda}

AWS Lambda to konkretna implementacja funkcji jako usługi oferowana przez Amazon Web Services od 2015 roku.
W ramach rozdziału opisano działanie usługi, wraz z jej najważniejszymi z perspektywy programisty cechami.
Skupiono się na konkretnym użyciu języka Java, który jest dostępny w serwisie.

AWS Lambda była pierwszą platformą FaaS oferowaną przez dużego dostawcę chmurowego.
Znacząco przyczyniło się to do popularyzacji paradygmatu bezserwerowego.
Usługa ta implementuje założenia modelu funkcji jako usługi, oferując programistom możliwość uruchamiania kodu bez konieczności zarządzania infrasktrukturą.
AWS Lambda przejmuje odpowiedzialność za przydzielenie zasobów, skalowanie, monitorowanie oraz zarządzanie cyklem życia wykonywanych funkcji,
Umożliwia to twórcom oprogramowania skupienie się na logice biznesowej aplikacji, rozliczając ich jedynie za faktyczny czas obliczeń i liczbę wywołań.

\begin{lstlisting}[caption={Przykładowa implementacja funkcji AWS Lambda w języku Java [źródło: opracowanie własne]} label={code:example_aws_lambda}]
public class MyHandler implements RequestHandler<MyHandler.Request, String> {
    public record Request(String message) {}

    public record Response() {}

    @Override
    public Response handleRequest(Request event, Context context) {
        // logika biznesowa funkcji
        return Response()
    }
}
\end{lstlisting}

Pierwszym krokiem wdrożenia funkcji AWS Lambda jest napisanie uruchamianego kodu.
Przykładowa implementacja została przedstawiona na [TODO: dodać listing]. 
Odpowiedni interfejs, który wykorzystuje programista, jest dostarczany przez AWS.
Jego implementacja w klasie MyHandler pozwala na nadpisanie metody handleRequest, która jest wywoływana w momencie obsługi zdarzenia.
W ciele metody umieszczana jest konkretna logika biznesowa funkcji chmurowej, na której skupia się użytkownik.

Aby kod został uruchomiony w chmurze należy utworzyć nową funkcję AWS Lambda.
Może to być wykonane poprzez konsolę Amazon Web Services, AWS CLI (ang. Command Line Interface) lub narzędzia infrasktruktury jako kod (np. AWS CDK, Terraform) \cite{awsLambdaDocs}.
Niezbędne do tego jest przygotowanie zasobów do wdrożenia.
Może to być wykonane na dwa sposoby: poprzez pliki ZIP lub JAR (istnieje także możliwość wdrożenia obrazów Dockerowych, jednak w ramach podrozdziału skupiamy się na środowisku uruchomieniowym Java) \cite{awsLambdaDocs}.
Tak przygotowany plik jest gotowy do wdrożenia podczas uruchamiania funkcji.

Kolejnym krokiem jest skonfigurowanie funkcji w usłudze AWS Lambda, co obejmuje zdefiniowanie kluczowych parametrów \cite{awsLambdaDocs}.
Jednym z najważniejszych jest pamięć funkcji, która odpowiada ilości pamięci RAM w megabajtach.
Innym parametrem jest limit czasu wykonania (timeout), który definiuje maksymalny czas, przez jaki pojedyncze wywołanie funkcji może być przetwarzane, zanim zostanie automatycznie zakończone przez platformę. 
Dodatkowo, programista ma możliwość zdefiniowania zmiennych środowiskowych, które pozwalają na przekazywanie konfiguracji do kodu funkcji w czasie jej działania.

Aby funkcja mogła zostać zintegrowana z tworzonym systemem należy skonfigurować źródła zdarzeń.
Zgodnie z naturą FaaS, funkcje Lambda są projektowane do uruchamiania w odpowiedzi na zdarzenia pochodzące z szerokiego ekosystemu usług AWS lub źródeł zewnętrznych.
Do typowych usług integrujących się z Lambda jako wyzwalacze należą m.in. \cite{awsLambdaDocs}:
\begin{itemize}
    \item Amazon S3 (np. przy tworzeniu nowego obiektu),
    \item Amazon API Gateway (dla żądań HTTP),
    \item Amazon DynamoDB Streams (reakcja na zmiany w tabelach),
    \item Amazon SQS (nowe wiadomości w kolejce),
    \item Amazon SNS (powiadomienia tematyczne),
    \item Amazon EventBridge (dla zdarzeń harmonogramowych oraz zdarzeń z innych usług i aplikacji).
\end{itemize}
Dzięki temu programista określa, w jakich okolicznościach i z jakimi danymi wejściowymi funkcja Lambda ma zostać automatycznie uruchomiona.

% Opisać środowiska wykonawcze

% Opisać cykl życia

% https://gemini.google.com/app/a27228310f848b08?hl=pl