\section{AWS Lambda}\label{chapter:aws_lambda}

AWS Lambda to konkretna implementacja funkcji jako usługi oferowana przez Amazon Web Services od 2015 roku.
W ramach rozdziału opisano działanie usługi, wraz z jej najważniejszymi z perspektywy programisty cechami.
Skupiono się na konkretnym użyciu języka Java, który jest dostępny w serwisie.

AWS Lambda była pierwszą platformą FaaS oferowaną przez dużego dostawcę chmurowego.
Znacząco przyczyniło się to do popularyzacji paradygmatu bezserwerowego.
Usługa ta implementuje założenia modelu funkcji jako usługi, oferując programistom możliwość uruchamiania kodu bez konieczności zarządzania infrasktrukturą.
AWS Lambda przejmuje odpowiedzialność za przydzielenie zasobów, skalowanie, monitorowanie oraz zarządzanie cyklem życia wykonywanych funkcji,
Umożliwia to twórcom oprogramowania skupienie się na logice biznesowej aplikacji, rozliczając ich jedynie za faktyczny czas obliczeń i liczbę wywołań.

\begin{lstlisting}[caption={Przykładowa implementacja funkcji AWS Lambda w języku Java [źródło: opracowanie własne]} label={code:example_aws_lambda}]
public class MyHandler implements RequestHandler<MyHandler.Request, String> {
    public record Request(String message) {}

    @Override
    public String handleRequest(Request event, Context context) {
        // logika biznesowa funkcji
    }
}
\end{lstlisting}

Pierwszym krokiem wdrożenia funkcji AWS Lambda jest napisanie uruchamianego kodu.
Przykładowa implementacja została przedstawiona na [TODO: dodać listing]. 
Odpowiedni interfejs, który wykorzystuje programista, jest dostarczany przez AWS.
Jego implementacja w klasie MyHandler pozwala na nadpisanie metody handleRequest, która jest wywoływana w momencie obsługi zdarzenia.
W ciele metody umieszczana jest konkretna logika biznesowa funkcji chmurowej, na której skupia się użytkownik.

% Pakowanie funkcji i upload do AWS Lambda

% Konfiguracja funkcji (typowe parametry np. pamięć)

% Konfiguracja wywołań (co możemy wybrać), jakie rodzaje

% Opisać środowiska wykonawcze

% Opisać cykl życia

% https://gemini.google.com/app/a27228310f848b08?hl=pl