\Chapter{Opis działania modelu FaaS}\label{chapter:wstep}
% Taki tytuł czy coś innego? Po polsku?

\section{Model serverless}\label{chapter:model_serverless}
Jednym z dynamicznie rozwijających się obszarów chmur obliczeniowych są usługi serverless. W 2025 roku rynek usług opartych o architekturę bezserwerową jest wart 17,88 miliarda dolarów amerykańskich, a według prognoz jego wartość wzrośnie do 41,14 miliarda w roku 2029~\cite{serverlessArchitectureMarketReport}. Badania wykonane na otwartoźródłowych projektach serverless wykazały, że architektura ta używana jest ze względu na niższe koszty, uproszczenie procesów operacyjnych (jak wdrażanie, skalowanie i monitorowanie) oraz bardzo wysoką skalowalność~\cite{ServerlessApplicationsWhyWhenAndHow}. Cechy te są osiągalne ze względu na wyjątkowe założenia tego modelu.

Przetwarzanie bezserwerowe możemy zdefiniować jako ,,formę przetwarzania w chmurze, która umożliwia użytkownikom uruchamianie aplikacji sterowanych zdarzeniami i rozliczanych granularnie bez konieczności zarządzania logiką operacyjną''~\cite{SpecRgCloudGroupVisionOnThePerformanceChallengesOfFaas}. W definicji tej znajdują się dwa ważne aspekty działania modelu bezserwerowego:

\begin{enumerate}
    \item ,,Przetwarzania w chmurze, która umożliwia użytkownikom uruchamianie aplikacji (\dots) bez konieczności zarządzania logiką operacyjną.''
    \item ,,Aplikacji sterowanych zdarzeniami i rozliczanych granularnie.''
\end{enumerate}

Pierwszy punkt odnosi się do zwiększenia zakresu odpowiedzialności dostawcy chmurowego w porównaniu do klasycznych usług (np. Amazon Elastic Compute Cloud). W usługach tych fizyczne serwery są utrzymywane przez dostawcę chmurowego, a użytkownik jedynie wynajmuje jednostki obliczeniowe. Posiada on dalej kontrolę nad konfiguracją wielu aspektów infrastruktury, co pozwala na większą wydajność rozwoju oprogramowania w porównaniu z środowiskami niechmurowymi. Mimo to, dalej wymaga to poświęcenia czasu i środków na skonfigurowanie oraz zabezpieczenie aplikacji. Architektury bezserwerowe mają na celu uproszczenie tych procesów. Podczas tworzenia aplikacji w usługach bezserwerowych zespoły programistyczne nie muszą zarządzać wdrożeniem, a następnie utrzymaniem serwerów (nawet w formie jednostek jak AWS EC2). Rolą twórcy oprogramowania jest dostarczenie kodu aplikacji lub obrazu Docker~\cite{awsLambdaDocs}~\cite{awsEcsDevGuide}, które zostaną uruchomione w utrzymywanym przez dostawcę chmurowego środowisku. Dzięki temu inżynierowie mogą skupić się w większym stopniu na logice aplikacji. Pozwala to na zmniejszenie liczby obowiązków, a co za tym idzie kosztów zespołu~\cite{riseOfThePlanetOfServerlessComputing}. 

Drugi punkt skupia się na charakterystycznym modelu płatności oraz sposobie działania architektur bezserwerowych, który umożliwia taki rodzaj rozliczeń. W przypadku klasycznych usług jak AWS EC2 płatność dokonywana jest za czas działania instancji, niezależnie od tego czy jest ona używana~\cite{awsEc2Guide}. Model ten 

% Tutaj wyjaśnić definicję:
% - "w chmurze" zatem tylko tam, bez naszych serwerów ale idziemy o krok dalej i niemusimy zarządzać np. wdrożeniem, skalowaniem, monitorowaniem itp.
% - "sterowanie zdarzeniami", że mamy event i on powoduje uruchomienie i rozliczane są za te uruchomienia zatem są tańsze 

