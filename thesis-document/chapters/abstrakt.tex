\section*{Streszczenie}

AWS Lambda to jedna z najważniejszych usług bezserwerowych oferowana przez Amazon Web Services.
Kluczowym aspektem tego typu serwisów jest wydajność, która przekłada się bezpośrednio na koszt użycia oraz doświadczenia użytkownika.
Zespoły programistyczne mogą tworzyć funkcje AWS Lambda z użyciem Javy oraz pokrewnych technologii z jej ekosystemu.
Narzędzia te mogą jednak powodować problemy z wydajnością, jak na przykład długi czas inicjalizacji.
Tworzy to potrzebę dostarczenia zespołom programistycznym odpowiednich metod poprawy wydajności.

Celem pracy było zaproponowanie nowych metod poprawy wydajności AWS Lambda dla rozwiązań z ekosystemu Java oraz analiza ich wpływu na wydajność funkcji i inne wybrane czynniki, które mogą wpłynąć na jakość pracy programistów.
W opracowaniu zawarto metody SnapStart, GraalVM, Kotlin JVM, Kotlin/JS oraz Kotlin/Native.

Przeprowadzono systematyczny przegląd literatury, który pozwolił na poznanie aktualnego stanu wiedzy o wydajności AWS Lambda.
Wpłynęło to na wybór metod oraz ostateczny kształt badania.
W pracy przeanalizowano wydajność funkcji pod kątem czasu wykonania podczas zimnych i ciepłych startów, zaproponowanego współczynnika wydajności funckji oraz kosztu działania.
Do badania wpływu na proces rozwoju oprogramowania użyto takżemetryk opisująch proces budowy artefaktu i ograniczeń w integracji z usługami zewnętrznymi.
W eksperymencie wzięto pod uwagę różne rozmiary pamięci funkcji.

Stwierdzono, że wybór metody optymalizacji powinien być zależny od charakterystyki funkcji.
Dla aktywnych funkcji z przewagą ciepłych startów najefektywniejsze okazały się GraalVM oraz Kotlin/Native.
Wykazano, że Kotlin/Native jest szczególnie skuteczny dla niewielkich rozmiarów pamięci funkcji, jak 128 MB oraz 256 MB.
Dodatkowo, cechuje się on najniższymi kosztami w tych wielkościach pamięci.

W pracy ustalono także kompromisy, z którymi muszą liczyć się zespoły programistyczne decydujące się na użycie poszczególnych metod.
Zalicza się do nich między innymi potencjalne ograniczenia w korzystaniu z funkcji języka, czas budowy artefaktu i ograniczenia w integracji z zewnętrznymi usługami.
Opracowanie dostarcza programistom wiedzy, który pozwala na świadomy wybór techniki poprawy wydajności funkcji.

\section*{Abstract}

Streszczenie  w języku angielskim.