\section{Kotlin/Native}\label{chapter:kotlin_native}

Bardzo skuteczną metodą optymalizacji wydajności może być rezygnacja z zarządzanych środowisk uruchomieniowych jak JVM.
Jedną z metod jest kompilacja do kodu natywnego, który uruchamiany jest bezpośrednio na systemie operacyjnym poza maszyną wirtualną.
Kotlin/Native oferuje zaawansową kompilację kodu Kotlin, który następnie może zostać uruchomiony w AWS Lambda z użyciem Amazon Linux.
W ramach rozdziału opisano sposób działania Kotlin/Native, jego możliwości i cechy, które mogą negatywnie wpłynąć na rozwój oprogramowania bezserwerowego.

Kluczowym elementami Kotlin/Native są kompilator oparty o LLVM oraz natywne implementacje bibliotek standardowych Kotlina \cite{kotlinlangKotlinDocs}.
Jak opisują Lattner i Adve \cite{1281665}, LLVM to framework kompilatora zaprojektowany do wspierania ciągłej i transparentnej analizy oraz transformacji programów.
Definiuje on wspólną, niskopoziomową reprezentację kodu w formie SSA (ang. Static Single Assignment) z systemem typów niezależnym od języka, co umożliwia implementację cech języków wysokiego poziomu. 
Głównym celem LLVM jest umożliwienie analizy i transformacji programu na różnych etapach jego życia, w tym w czasie kompilacji, linkowania, uruchomienia oraz w czasie bezczynności między uruchomieniami.
Jego użycie pozwala następnie na zbudowanie natywnych plików binarnych, które mogą zostać uruchomione bezpośrednio na docelowej platformie, dla której zostały skompilowane.
Wymaga to jednak dokładnego określenia systemu operacyjnego i architektury już w momencie kompilacji.

Kompilacja Kotlina do kodu natywnego otwiera przez KMP możliwość tworzenia samodzielnych programów wykonywalnych, które nie wymagają zewnętrznego środowiska uruchomieniowego.
Znajduje to zastosowanie w scenariuszach takich jak rozwój aplikacji mobilnych na platformę iOS, współdzielenie logiki biznesowej między różnymi platformami (np. Android i iOS) czy budowa narzędzi konsolowych.
Dlatego ważnym aspektem Kotlin/Native jest współdziałanie z istniejącym kodem natywnym.
Pozwala to na bezpośrednie wywoływanie funkcji z bibliotek napisanych w języku C, a na platformach firmy Apple również Objective-C \cite{kotlinlangKotlinDocs}.
Znacząco rozszerza to zakres dostępnych narzędzi i bibliotek, które mogą zostać użyte przez programistów.

Kotlin/Native oferuje wiele różnych platform docelowych, rozszerzając tym samym zakres zastosowań języka Kotlin.
Wśród wspieranych systemów docelowych znajdują się platformy Apple (takie jak macOS, iOS, watchOS, tvOS), Android, a także systemy z rodziny Windows oraz Linux \cite{kotlinlangKotlinDocs}.
Szczególnie istotnie dla usługi AWS Lambda są jednak platformy linuxowe.
Są to linuxX64 oraz linuxArm64, które pozwalają na uruchomienie kodu z użyciem odpowiednio architektur x86 oraz ARM.
Pozwala to następnie na ich bezpośrednie użycie w AWS Lambda, działającej bezpośrednio w systemie Amazon Linux.
Dzięki temu możliwa jest poprawa wydajności, szczególnie w aspekcie zimnych startów, które są znacznym wyzwaniem dla funkcji opartych o JVM.

Istotnym elementen funkcji bezserwerowych jest zarządzanie pamięcią, która wpływa bezpośrednio na koszty.
W przypadku Kotlin/Native, ewolucja modelu zarządzania pamięcią znacząco wpłynęła na jego użyteczność i możliwości optymalizacyjne.
Początkowa technologia ta opierała się na restrykcyjnym modelu z izolacją obiektów między wątkami \cite{kotlinlangKotlinDocs}.
Powodowało to skomplikowane zarządzanie stanem w operacjach współbieżnych.
Aktualnie, Kotlin/Native implementuje nowy menedżer pamięci.
Wprowadza on automatyczne zarządzanie pamięcią poprzez współbieżny, nieblokujący moduł zbierania śmieci (ang. garbage collector) \cite{kotlinlangKotlinDocs}.
Znacząco upraszcza to programowanie współbieżne, które teraz nie wymaga ręcznego zarządzania obiektami.
Mechanizm ten wprowadza intuicyjne współdzielenie stanu, które jest analogiczne do środowiska JVM, jednak bez konieczności kosztownej obsługi maszyny wirtualnej Java.

Mimo potencjalnych zysków w ramach wydajności, użycie Kotlin/Native może nieść utrudnienia w kontekście integracji z bibliotekami zewnętrznymi.
Język C nie jest oficjalnie wspierany jako język programowania AWS Lambda \cite{awsLambdaDeveloperGuide}, co wynika zapewne z jego niewielkiej popularności na tej platformie.
Współdziałanie z kodem natywnym w Kotlin/Native skupia się jednak na platformach klienckich, co nie musi być do końca użyteczne w zakresie AWS Lambda.
Amazon Web Services oferuje swoje SDK w języku C++, który nie jest jednak łatwo integrowalny z Kotlin/Native (w odróżnieniu od C i Objective-C).
Jest to ważny czynnik, który powinien być uwzględniony przez programistów projektujących aplikacje bezserwerowe.

% Plan
% 0. Wstęp

% 1. Jak to działa:
% - Opisać LLVM
% - Gdzie się głównie używa
% - Wspomnieć o interoperacyjności z C, Objective-C itp
% - Jakie platformy? (https://kotlinlang.org/docs/native-target-support.html#for-library-authors)

% 2. Zarządzanie pamięcią (https://kotlinlang.org/docs/native-memory-manager.html)

% 3. Wady - problem z dostępnością bibliotek
% - AWS SDK może poprzez C++ jednak może to wymagać większego nakładu pracy
