\chapter{Wybrane metody optymalizacji}\label{chapter:wybrane_metody_optimalizacji}

W ramach rozdziału dokonano charakteryzacji wybranych metod optymalizacji, które zostały zawarte w pracy.
Wybór metod podyktowany był wynikami przeglądu literatury, na podstawie którego do badań włączono metody SnapStart i GraalVM.
Według analizowanych badań techniki te były skuteczne, jednak ich reprezentacja w dotychczasowych publikacjach była niewielka.
Kluczowym elementem rozdziału są nowe metody optymalizacji, które zostały zaproponowane w ramach pracy.
Strategie te opierają o język Kotlin, który wybrano ze względu na brak prac poruszających inne języki ekosystemu Java niż sama Java (np. Kotlin, Scala, Groovy).
Kotlin cechuje się różnymi potencjalnymi zastosowaniami, na bazie których wybrano trzy podejścia, które zostały zaproponowane jako nowe metody optymalizacji.
W opisie każdej metody przedstawiono sposób jej działania, a także jej zalety i wady, co pomoże w odpowiedzi na trzeci pytanie badawcze sformułowane w pracy.

\section{SnapStart}\label{chapter:snapstart}

Jednym z istotnych czynników wpływających na wydajność funkcji AWS Lambda implementowanych w ekosystemie Java jest zjawisko tzw. zimnego startu.
Wynika on z cyklu życia funkcji i etapu inicjalizacji (co zostało opisane w Rozdziale \ref{chapter:aws_lambda}).
W etap ten wchodzą procesy takie jak inicjalizacja maszyny wirtualnej Java czy uruchomienie statycznego kodu inicjującego \cite{awsLambdaDocs}.
W przypadku Javy zajmuje to więcej czasu niż dla innych języków (jak Python), co wydłuża zimne starty \cite{8605773}.
Znacząco oddziałuje to na wydajność funkcji, a może być szczególnie dotkliwe dla serwisów o niewielkiej aktywności.
W odpowiedzi na potrzebę minimalizacji tych negatywnych skutków Amazon Web Services wprowadziło mechanizm znany jako AWS Lambda SnapStart.
W ramach podrozdziału podjęto analizę tego rozwiązania w kontekście jego działania, zalet oraz ograniczeń. 

Mechanizm SnapStart istotnie modyfikuje tradycyjny cykl życia funkcji AWS Lambda.
Zasadnicza różnica polega na przeniesieniu kosztownego etapu inicjalizacji z momentu pierwszego wywołania funkcji na etap jej publikacji \cite{amazonSnapstartDeveloperGUide}.
Oznacza to, że inicjalizacja funkcji nie jest wykonywana w momencie zapytania użytkownika (co wywołuje zimny start), lecz w momencie wgrania nowej wersji funkcji (oraz kodu) przez programistę.
Inicjalizacja ta zawiera najdłuższe operacje dla rozwiązań Java jak utworzenie maszyny wirtualnej, załadowanie klas, czy wykonanie kodu inicjalizującego.
Następnie, tworzona jest zaszyfrowana ,,migawka'' (ang. snapshot) stanu pamięci i dysku w pełni gotowego środowiska wykonawczego.
Gdy funkcja jest następnie wywoływana po raz pierwszy, nie zachodzi już standardowy zimny start.
Zamiast tego środowisko jest odtwarzane z utworzonej migawki, co zostało przedstawione na Rysunku \ref{fig:aws_lambda_snapstart_process}.
Według dostawcy AWS metoda ta w optymalnych scenariuszach zmniejsza opóźnienie z kilku sekund do mniej niż sekundy \cite{amazonSnapstartDeveloperGUide}. 

\begin{figure}[h]
    \centering
    \includegraphics[width=0.95\textwidth]{charts/snapstart.png}
    \caption{Proces uruchomienia AWS Lambda z użyciem metody SnapStart [źródło:~opracowanie~własne]}
    \label{fig:aws_lambda_snapstart_process}
\end{figure}

Działanie tej metody jest technicznie możliwe dzięki użyciu środowiska wirtualizacji przez AWS Lambda.
Jak opisują Agache i inni autorzy \cite{246288} usługa Lambda do izolacji poszczególnych funkcji wykorzystuje dedykowane maszyny wirtualne typu microVM.
Są one zarządzane przez lekki monitor maszyn wirtualnych (ang. VMM) o nazwie Firecracker.
Posiada on cechy, które były kluczowe w minimalizacji problemu zimnego startu.
Po pierwsze, celowo rezygnuje on z emulacji zbędnych urządzeń (jak emulacja systemu BIOS czy rozbudowanych kontrolerów PCI) \cite{246288}.
Zmniejsza to złożoność i rozmiar stanu każdej maszyny wirtualnej. 
Dzięki temu wykonanie i odtworzenie migawki jest łatwiejsze.
Po drugie, Firecracker jest w pełni kontrolowany przez interfejs REST API \cite{246288}.
Umożliwia to precyzyjne zarządzanie całym cyklem życia każdej maszyny wirtualnej, włączając w to jej konfigurację, uruchomienie oraz zatrzymanie.
Pozwala to na określenie fazy inicjalizacji funkcji oraz wykonanie migawki w odpowiednim momencie.
Istotna jest również zapewniona przez Firecracker izolacja \cite{246288}, co gwarantuje bezpieczeństwo tworzenia i odtwarzania migawek.

Samo użycie metody SnapStart jest bardzo proste i nie wymaga od programisty dużego nakładu pracy.
Włączenie rozwiązania wymaga jedynie ustawienia odpowiedniej opcji podczas konfiguracji funkcji \cite{amazonSnapstartDeveloperGUide}.
Nie oznacza to jednak, że SnapStart jest odpowiedni dla wszystkich funkcji.
AWS podkreśla dwa typy aplikacji, które znacząco zyskają poprzez użycie SnapStart \cite{amazonSnapstartDeveloperGUide}.
Są nimi wrażliwe na opóźnienia interfejsy API i potoki przetwarzania danych.
Dodatkowo, metoda ta niesie za sobą pewne ograniczenia, które muszą zostać uwzględnione przed jej wdrożeniem.

Pierwszym aspektem jest kwestia unikalności stanu w funkcjach wykorzystujących SnapStart.
Jak analizują Brooker i inni autorzy \cite{brooker2021restoringuniquenessmicrovmsnapshots}, klonowanie migawek wprowadza fundamentalne wyzwanie związane z przywróceniem unikalności maszyn wirtualnych, co jest niezbędne do poprawnego generowania unikalnych identyfikatorów czy sekretów kryptograficznych.
Migawka zainicjowanego środowiska wykonywana jest jednorazowo, a następnie używana podczas wielu wywołań funkcji.
Może to stanowić duże zagrożenie dla programisty AWS Lambda, gdy potrzebuje on generować unikalne wartości jak identyfikatory (np. UUID) czy jednorazowe tokeny.
Narusza to znacznie poprawność logiki aplikacji oraz jej bezpieczeństwo.
Jedną z metod naprawy tego problemu jest generowanie wartości losowych wyłącznie w metodzie wywołującej funkcje (zamiast w bloku statycznym kodu) \cite{amazonSnapstartDeveloperGUide}.
Dodatkowo, ewentualne problemy z unikalnością funkcji SnapStart mogą zostać wykryte poprzez oprogramowanie SpotBugs \cite{SpotBugsProject}.
Narzędzie wykonuje statyczną analizę kodu, walidując go poprzez reguły zapewnione przez AWS.
Pozwala to programiście wykryć, a następnie naprawić fragmenty kodu powodujące problem z unikalnością.

Kolejnym istotnym wyzwaniem podczas rozwoju aplikacji z technologią SnapStart jest zarządzanie połączeniami sieciowymi \cite{amazonSnapstartDeveloperGUide}.
Połączenia nawiązane z zewnętrznymi usługami są standardową praktyką podczas tworzenia aplikacji AWS Lambda \cite{eismann2021reviewserverlessusecases}\cite{Ivanov_Petrova_2024}.
Problematyczne stają się jednak te połączenia sieciowe, które nawiązano podczas inicjalizacji funkcji. 
Ponieważ inicjalizacja odbywa się przed faktycznym przetworzeniem żądania użytkownika, upływający czas może sprawić, że w momencie odtworzenia funkcji połączenia te nie będą już aktywne.
Praktyką zalecaną przez AWS jest ponowne nawiązywanie lub dokładna walidacja istniejących połączeń \cite{amazonSnapstartDeveloperGUide}.
Powinno to być wykonane bezpośrednio w metodzie wywołującej funkcje lub z wykorzystaniem metody ,,afterRestore''.
Metoda ta jest wywoływana bezpośrednio po odtworzeniu migawki stanu funkcji.

Strategicznym czynnikiem usług bezserwerowych są koszty, zatem powinny być one uwzględnione także przed użyciem SnapStart.
Zgodnie z dokumentacją Amazon Web Services \cite{amazonSnapstartDeveloperGUide}, użycie SnapStart dla środowisk uruchomieniowych Java nie wiążą się z dodatkowymi kosztami.
Koszt wykonania funkcji z włączonym SnapStart nadal bazuje na standardowych rozliczeniach.
Składa się na niego liczba przetworzonych żądań oraz łączny czas trwania wykonań.

Podsumowując, mechanizm SnapStart stanowi prostą w aktywacji metodę redukcji czasu zimnych startów.
Sam mechanizm opiera się na wcześniejszym wykonaniu fazy inicjacji funkcji, a następnie wykonania migawki stanu.
W momencie wywołania funkcji stan ten może zostać odtworzony.
Znaczącą korzyścią metody jest brak dodatkowych kosztów.
Wiąże się ona jednak z istotnymi utrudnieniami (jak zarządzanie połączeniami sieciowymi i problem z unikalnością stanu).
Powinny być one uwzględnione przez programistę przed użyciem narzędzia.
\section{GraalVM}\label{chapter:graalvm}

Ważnym obszarem badań nad optymalizacją Javy i jej użycia w AWS Lambda, są technologie pozwalające na zmianę sposobu kompilacji i uruchamiana aplikacji.
Jedną z technologii, które zyskusje na popularności w tym zakresie, jest GraalVM.
Jest to możliwe m.in. dzięki użyciu kompilatora JIT (ang. Just-In-Time) w połączeniu z kompilacją AOT (ang. Ahead-Of-Time) \cite{8756917}.
GraalVM oferuje zaawansowaną architekturę pozwalającą na kompilację i uruchomienie aplikacji w postaci obrazów natywnych.
Stanowi to alternatywę dla klasycznej maszyny wirtualnej Javy, a dodatkowo skupia się na jej wydajności.
Poniższy podrozdział poświęcono analizie działania omawianego rozwiązania, jego zalet i słabych stron.

Kluczowym mechanizmem GraalVM jest kompilacja AOT (ang. Ahead-Of-Time) do postaci tzw. obrazów natywnych (ang. native images) \cite{8756917}.
Ma to bezpośredni wpływ na wydajność działania aplikacji.
W modelu tradycyjnym, kod bajtowy Java jest interpretowany i kompilowany dynamicznie przez maszynę wirtualną w trakcie działania aplikacji.
Podejście AOT przenosi znaczną część z tych operacji na etap budowania artefaktu. 
Istotnym elementem tego procesu jest agresywna, statyczna analiza kodu \cite{9245290}, w celu identyfikacji osiągalnych w trakcie działania części.
Pozwala to na odrzucenie nieużywanych fragmentów kodu (np. z używanych bibliotek), co pozwala na zmniejszenie wielkości obrazu natywnego.
Aspekt ten może być kluczowy w kontekście AWS Lambda, ze względu na wpływ wielkości artefaktu na wydajność \cite{8116416}.
Po analizie kodu, dokonywana jest inicjalizacja klas, a stan aplikacji, w tym częściowo zainicjalizowana sterta, jest utrwalany.
W celu lepszej optymalizacji, operacje te są powtarzane, co zostało przedstawione na Rysunku \ref{fig:graalvm_build_process}.

Jako wynik kompilacji powstaje samodzielny, zoptymalizowany plik binarny.
Nie wymaga on do uruchomienia pełnej maszyny wirtualnej Java, a jedynie minimalnego środowiska wykonawczego dostarczanego przez SubstrateVM, będącego częścią GraalVM \cite{8756917}.
Różnica ta ma fundamentalne znaczenie w kontekście wydajności AWS Lambda.
Eliminowana jest konieczność wykonywania czasochłonnych operacji typowych dla startu tradycyjnej maszyny wirtualnej Java, takich jak ładowanie klas czy jej inicjalizacja.
Wszystkie te zadania zostały już wykonane wcześniej, w procesie budowy obrazu natywnego.
Dzięki temu tworzona przez programistę funkcja AWS Lambda nie będzie operować w zarządzanym środowisku Java.
Zamiast tego, usługi muszą opierać się o niestandardowe środowiska wykonawcze, oferujące wyłącznie system operacyjny (Amazon Linux 2023 lub Amazon Linux 2) \cite{awsLambdaDeveloperGuide}.
Ich użycie pozwala także na realizację drugiej zalety GraalVM, czyli redukcji zapotrzebowania na pamięć operacyjną \cite{9245290}.

Pomimo pozytywnego wpływu na wydajność, zastosowanie kompilacji AOT w GraalVM wiąże się także z ograniczeniami.
Jednym z nich jest obsługa dynamicznych cech Javy, takich jak refleksja (ang. reflection), dynamiczne proxy, serializacja czy natywny interfejs Java (JNI)
Wynika to z faktu użycia agresywnej statycznej analizy kodu.
Napotyka ona trudności w przewidzeniu wszystkich dynamicznie ładowanych klas, pól i metod, które nie są jawnie osiągalne w kodzie źródłowym.
Problem ten wymaga użycia dodatkowych mechanizmów GraalVM \cite{graalvm-reflection-jdk21}.
Polegają one na przygotowaniu dodatkowych metadanych dla klas, co wymaga jednak dodatkowej obsługi.

\begin{figure}
    \centering
    \includegraphics[width=0.95\textwidth]{charts/graalvm-build-process.drawio.png}
    \caption{Uproszczony proces budowy obrazu natywnego GraalVM [źródło:~opracowanie~własne]}
    \label{fig:graalvm_build_process}
\end{figure}

Kolejnym aspektem, który może negatywnie wpłynąć na rozwój oprogramowania przy użyciu GraalVM, jest czasochłonność procesu kompilacji.
Generowanie w pełni zoptymalizowanego obrazu natywnego jest operacją bardziej złożoną niż standardowa kompilacja kodu Javy do postaci bajtowej.
W praktyce oznacza to, że proces budowania artefaktu dla funkcji AWS Lambda może trwać odczuwalnie dłużej.
Może mieć to znaczący wpływ na rozwój oprogramowania, szczególnie w przypadku częstych iteracji i tworzenia nowych wersji funkcji.
Dłuższy czas kompilacji może także wpłynąć na ogólną efektywność procesów ciągłej integracji i ciągłego dostarczania (ang. CI/CD).

Jednym z sposobów poprawy doświadczeń programistów przy pracy z GraalVM, jest użycie odpowiednich frameworków.
Jednym z nich jest Quarkus \cite{9245290}, który został zaprojektowany z myślą o środowiskach chmurowych.
Kluczową cechą Quarkusa jest przeniesienie jak największej liczby operacji inicjalizacyjnych i konfiguracyjnych na etap budowania aplikacji.
Obejmuje to między innymi wstrzykiwanie zależności, przetwarzanie adnotacji oraz konfigurację rozszerzeń. 
Dzięki temu, w czasie budowania obrazu natywnego, Quarkus jest w stanie przeprowadzić szczegółową analizę aplikacji.
Poprzez użycie odpowiednich annotacji pozwala on na oznaczenie klas niezbędnych dla mechanizmów refleksji czy proxy \cite{quarkus-docs}.
Dzięki temu jest on w stanie automatycznie wygenerować niezbędne metadane dla klas.
Dane te następnie pozwalają na użycie wspomnianych mechanizmów w GraalVM.
Innymi, konkurencyjnymi do Quarkusa frameworkami, które oferują wsparcie dla obrazów natywnych są Helidon i Micronaut \cite{9245290}.
Ich popularność wskazuje na wysokie zainteresowanie takimi technologiami w społeczności programistów Java, dlatego jest to interesujący kierunek rozwoju dla funkcji AWS Lambda.

\section{Kotlin}\label{chapter:kotlin_multiplatform}

W ramach systematycznego przeglądu literatury (przedstawionego w Rozdziale \ref{chapter:przeglad_literatury}) zauważono, że aktualne badania skupiają się wyłącznie na języku Java.
Pomijają one jednak inne języki z ekosystemu Java, także oparte o maszynę wirtualną Java (ang. JVM).
Tymczasem na popularności zyskują alternatywne języki ekosystemu Javy.
Wśród nich na szczególną uwagę zasługuje Kotlin, rozwijany przez firmę JetBrains.
Jest on oficjalnie wspierany przez Google jako język programowania dla platformy Android, co wskazuje na jego solidne zastosowanie w tych systemach.
Coraz większe uznanie zyskuje także jako działający po stronie serwera. 
Dlatego język ten jest interesującym obszarem badań w kontekście AWS Lambda.
W tym rozdziale przedstawiony zostanie język programowania Kotlin, w tym jego zastosowanie w kontekście funkcji AWS Lambda.

Jednym z kluczowych czynników, które zwiększają popularność Kotlina, jest jego łatwa nauka przez programistów Javy.
Dodatkowo, istnieje możliwość łatwej intergracji kodu napisanego w Kotlinie z istniejącym oprogramowaniem Java \cite{kotlinlangKotlinDocs}.
Te cechy czynią go interesującym kandydatem do analizy w kontekście optymalizacji wydajności rozwiązań dla usługi AWS Lambda. 
Dla zespołów programistycznych może stanowić on wartościowe rozszerzenie dotychczasowych możliwości. 
Kotlin oferuje bowiem alternatywę lub uzupełnienie dla tradycyjnie stosowanej Javy.

Kwestia wydajności Kotlina w porównaniu do Javy jest przedmiotem dyskusji. 
Jednak badania dotyczą najczęściej ich zastosowań w kontekście aplikacji mobilnych.
Gajek i inni autorzy \cite{Gajek_Plechawska-Wójcik_2024} przeanalizowali wydajność obu języków, poprzez użycie gry mobilnej uruchomionej na systemie Android.
Wykazali oni, że w testowanym scenariuszu Java osiągnęła nieznacznie lepszą wydajność pod względem zużycia zasobów CPU i RAM.
Było to jednak zastosowanie mobilne, a same różnice nie były znaczne.
Należy jednak podkreślić, że warunki mobilne mogą być inne niż w systemach działających w usłudze AWS Lambda.
Sam język Kotlin posiada mechanizmy, które mogą pozytywnie wpłynąć na wydajność.

Funkcje inline (ang. inline functions) w Kotlinie mogą przyczynić się do redukcji narzutu wydajności podczas wywołań funkcji.
Mechanizm ten polega na wstawieniu kodu ciała funkcji bezpośrednio w miejsce jej wywołania \cite{kotlinlangKotlinDocs}.
Jest to wykonywane w momencie kompilacji, a programista może określić, które funkcje powinny być w ten sposób optymalizowane.
Eliminuje to koszt ich wywołania, co jest szczególnie przydatne w przypadku małych, często używanych funkcji.
Dodatkowo, język pozwala na przekazywanie funkcji jako parametrów, na przykład w kolekcjach i metodach jak filtrowanie.
W tych sytuacjach użycie funkcji inline może znacząco zmniejszyć liczbę operacji.
Pozytywny wpływ mechanizmu inline został przedstawiony przez Bergstrom i innych autorów \cite{DBLP:journals/corr/BergstromFRS13}, gdzie jego użycie zmniejszyło czas wykonywania programów nawet do 8\%.

Innym istotnym elementem Kotlina wspierającym wydajność są korutyny (ang. coroutines).
Mogą być one użyteczne zwłaszcza w kontekście operacji wejścia-wyjścia (I/O).
Systemy oparte o usługę AWS Lambda często integrowane są z zewnętrznymi serwisami (co zostało zauważone w ramach przeglądu literatury w Rozdziale \ref{chapter:przeglad_literatury}).
Wymaga to komunikacji opartej o operacje sieciowe.
Tradycyjne podejście oparte na wątkach może konsumować dużą ilość zasobów serwera i prowadzić do blokowania wykonania.
Korutyny pozwalają na pisanie asynchronicznego, nieblokującego kodu w sposób bardziej sekwencyjny i czytelny \cite{kotlinlangKotlinDocs}.
Na lepszą wydajność korutyn w porównaniu z tradycyjnymi wątkami wskazali Beronić i inni autorzy \cite{9803765}.

Implementacja mechanizmów poprawiających wydajność w języku programowania, pozwala następnie na ich użycie w bibliotekach, które są wykorzystywane przez programistów.
Język Kotlin oferuje ciekawy ekosystem bibliotek, przeznaczonych na przykład do tworzenia aplikacji działających po stronie serwera.
Są to biblioteki jak http4k czy ktor.
Ktor to framework zaprojektowany do budowy asynchronicznych aplikacji serwerowych i klienckich, rozwijany przez firmę JetBrains.
Jest on oparty w pełni o język Kotlin, a jego kluczową cechą jest natywne wsparcie dla korutyn.
Z kolei http4k kładzie nacisk na prostotę i minimalizm.
Architektura http4k opiera się na koncepcji funkcji jako podstawowych bloków aplikacji \cite{http4kCoreDocs}, co naturalnie współgra z modelem serverless i AWS Lambda.
Samo narzędzie rezyguje z mechanizmów refleksji \cite{http4kCoreDocs}, co może mieć pozytywny wpływ na wydajność.

Rosnące znaczenie Kotlin dostrzega także Amazon Web Service, które oferuje bibliotekę AWS SDK dla Kotlina \cite{awsSDKForKotlinDeveloperGuide}.
Jej celem jest zapewnienie programistom możliwości interakcji z usługami AWS w sposób naturalny dla tego języka.
SDK ten został zaprojektowany od podstaw z myślą o Kotlinie, co przejawia się między innymi w wykorzystaniu korutyn do obsługi operacji asynchronicznych.

Duży wpływ na wydajność funkcji AWS Lambda ma wybrany język programowania \cite{8605773}\cite{Cordingly2020704}.
Wynika to na przykład z różnych przypadków biznesowych i operacji, które muszą wykonywać.
Mimo to, często muszą one dzielić wspólny kod \cite{8116416}, co wskazuje na potrzebę wykorzystania mechanizmów, które to umożliwą.
Z tego powodu bardzo interesującą dla AWS Lambda i jej wydajności, może okazać się inicjatywa Kotlin Multiplatform.
KMP (Kotlin Multiplatform) to projekt, który powstał w szczególności dla aplikacji mobilnych.
Pozwala on na kompilację lub translację tego samego kodu Kotlin do użycia na różnych platformach.
Mogą to być na przykład Android, iOS, aplikacje desktopowe (JVM) lub webowe (JavaScript, Web Assembly) \cite{kotlinlangKotlinDocs}.
Oferuje to możliwość dzielenia kodu (np. logiki biznesowej) pomiędzy różnymi platformami, jednak przy możliwości zachowania natywnych komponentów widoku.

Mimo głównego przypadku użycia jakim są aplikacje klienckie, Kotlin Multiplatform może być obiecującym rozwiązaniem dla AWS Lambda.
Po pierwsze oferuje on możliwość translacji kodu Kotlin do JavaScript oraz kompilację do natywnych plików binarnych (opcje te zostaną przedstawione jako osobne metody w kolejnych rozdziałach).
Pozwala to na ominięcie różnych niedogodności wynikających z użycia maszyny wirtualnej Javy.
Jednak zachowane są przy tym zalety języka oraz wspiera to użycie już istniejących umiejętności programistów języków rodziny JVM.
Po drugie, kod w KMP może być dzielony pomiędzy platformami.
Umożliwia to bardzo elastyczny wybór środowiska uruchomieniowego AWS Lambda w ramach pojedynczego systemu.
Jednocześnie, część kodu może być współdzielona między wszystkie funkcje niezależnie od wybranej platformy.
Może to na przykład oznaczać, że klasy implementujące pewne struktury oraz zasady wynikające z reguł biznesowych, będą mogły być używane przez funkcje działające zarówno poprzez JavaScript, JVM, jak i natywne pliki binarne. 

Jednym z czynników, które mogą być modyfikowane już podczas działania usług AWS Lambda jest pamięć.
Jej rozmiar może być dostosowywany w zależności od wydajności monitorowanej funkcji.
Użycie Kotlin Multiplatform pozwala na rozszerzenie tej metody.
W zależności od obserwowanych parametrów (jak czas odpowiedzi lub opóźnienia zimnych startów) możliwe jest ponowne wykorzystanie tego samego kodu i budowa funkcji działającej na innej platformie.
Przykładowo, po wdrożeniu funkcji działającej z użyciem JVM, może pojawić się potrzeba redukcji czasu zimnych startów.
W takim wypadku Kotlin Multiplatform umożliwia translację kodu do JavaScript, który pozwoli na redukcję czasu inicjalizacji AWS Lambda.

Współdzielenie kodu pomiędzy platformami jest możliwe dzięki strukturze, którą oferuje Kotlin Multiplatform.
Została ona opisana przez firmę JetBrains w ramach dokumentacji KMP \cite{kotlinMultiplatformDev}.
Jej głównym elementem jest katalog ,,commonMain'', który jest współdzielony pomiędzy wszystkimi platformami.
Kompilator używa kodu współdzielonego jako dane wejściowe, aby w rezultacie utworzyć zestaw plików binarnych specyficznych dla danej platformy.
Mogą to być na przykład pliki .class dla maszyny wirtualnej Javy, czy natywne pliki wykonywalne (np. dla platformy Linux).

\begin{figure}[h]
    \centering
    \includegraphics[width=0.95\textwidth]{charts/kmp-structure.drawio.png}
    \caption{Przykładowa struktura pojektu Kotlin Multiplatform [źródło:~opracowanie~własne]}
    \label{fig:kmp_project_structure}
\end{figure}

Następnie programista może utworzyć kolejne katalogi, które będą zawierać kod specyficzny dla docelowych platform.
Przykładowa struktura została przedstawiona na Rysunku \ref{fig:kmp_project_structure}, gdzie katalog commonMain jest wspóldzielony między JVM (jvmMain), JavaScript (jsMain) oraz platformy natywne (nativeMain).
Docelowe platformy (ang. targets) są deklarowane w konfiguracji Gradle \cite{kotlinMultiplatformDev}, a kod współdzielony jest przygotowywany wyłącznie dla nich.
Katalogi dla docelowych platform są wymagane, gdyż Kotlin nie zezwala na użycie specyficznych elementów danej platformy w katalogu współdzielonym.
Przykładem takiego elementu może być klasa ,,java.io.File'', która jest dostępna wyłącznie dla maszyny wirtualnej Javy.
Jej użycie w katalogu commonMain spowoduje błąd kompilacji.

Kotlin Multiplatform zawiera także integracje z testami oprogramowania.
Jest to szczególnie ważne dla tworzenia oprogramowania z wykorzystaniem AWS Lambda, gdzie testowanie może być skomplikowane (co było jednym z wniosków przeglądu literatury w Rozdziale \ref{chapter:przeglad_literatury}).
Testy dla kodu współdzielonego powinny być zapisane w katalogu ,,commonTest'', gdzie programista może użyć biblioteki kotlin.test \cite{kotlinMultiplatformDev}.
Następnie testy są uruchamiane dla każdej docelowej platformy.
Programista może także tworzyć przypadki testowe dla konkretnych platform, z użyciem technologii przez nie oferowanych.
Następuje tutaj analogiczne współdzielenie kodu jak dla katalogów ,,main'', co zostało także zawarte w Rysunku \ref{fig:kmp_project_structure}.

Specyficzne cechy Kotlina jak funkcjonalności języka, biblioteki czy projekt Kotlin Multiplatform mogą zapewnić znaczne wzrosty wydajności funkcji AWS Lambda.
Mimo, że Kotlin jest językiem wywodzącym się z Javy oferuje już możliwości, które mogą pozwolić na osiągnięcie niższych czasów odpowiedzi.
Dodatkowo, sposoby te nie zostały jeszcze przebadane. 
Dlatego Kotlin to obszar, który zasługuje na zawarcie go w badanich na temat wydajności AWS Lambda.

\section{Kotlin/JS}\label{chapter:kotlin_js}

Podczas wyboru języka programowania działającego w AWS Lambda częstym kryterium może być czy język jest kompilowany, czy interpretowany.
Kotlin to domyślnie język kompilowany, jednak poprzez projekt Kotlin Multiplatform zapewnia on możliwość translacji do języka JavaScript.
Jednocześnie zachowuje on możliwość używania Kotlina, który jest łatwy do nauki przez programistów Java.
W ramach rozdziału opisano sposób działania Kotlin/JS oraz wybrane cechy i ograniczenia, które wpływają na użycie w AWS Lambda.

Projekt Kotlin Multiplatform opiera się na możliwości wskazania docelowych platform dla kompilatora Kotlin, aby kod napisany w tym języku mógł być używany w środowiskach innych niż maszyna wirtualna Java.
Podobnie jest w przypadku Kotlin/JS, który jest jedną z dostępnych docelowych platform KMP \cite{kotlinlangKotlinDocs}.
Cały proces opiera się na translacji między językami, która nie jest jednak bezpośrednia.
Dodatkowym krokiem jest użycie reprezentacji pośredniej Kotlina (ang. Kotlin intermediate representation, IR) \cite{kotlinlangKotlinDocs}.
Po pierwsze, kod źródłowy Kotlina jest transformowany do reprezentacji pośredniej.
Następnie reprezentacja ta jest stopniowo kompilowana do kodu JavaScript, który może zostać uruchomiony na docelowej platformie.
Proces transformacji został przedstawiony na Rysunku \ref{fig:kotlin_js_ir}. 
Każdy wspierany przez kompilator IR element języka Kotlin jest reprezentowany w formie reprezentacji pośredniej.
Mogą to być m.in. funkcje (wraz ich argumentami, typami oraz modyfikatorami dostępu), instrukcje ,,return'', czy instrukcje warunkowe wraz z wszystkimi rozgałęzieniami. 
W kolejnym etapie, dla poszczególnych elementów reprezentacji pośredniej, dobierane są ich odpowiedniki właściwe dla docelowej platformy, co w przypadku Kotlin/JS skutkuje wygenerowaniem kodu JavaScript.

\begin{figure}[h]
    \centering
    \includegraphics[width=1\textwidth]{charts/kotlin-js-ir.drawio.png}
    \caption{Proces transformacji kodu Kotlin do kodu JavaScript [źródło:~opracowanie~własne na bazie repozytorium kompilatora IR \cite{kotlinIrCompilatorGithub}]}
    \label{fig:kotlin_js_ir}
\end{figure}

Istotnym elementem Kotlin/JS jest wsparcie dla dwóch środowisk wykonawczych: Node.js oraz przeglądarkowego.
W przypadku przeglądarek Kotlin oferuje wsparcie dla DOM API (ang. Document Object Model) \cite{kotlinlangKotlinDocs}, co pozwala na łatwiejszy rozwój aplikacji klienckich działających w przeglądarce.
Ważniejsze dla AWS Lambda jest jednak wsparcie dla Node.js, który jest środowiskiem wykonawczym wspieranym przez usługę.
Wybór środowiska jest łatwo określany poprzez konfigurację narzędzia budującego (np. Gradle) \cite{kotlinlangKotlinDocs}.

Język JavaScript jest znacznie wspierany przez liczną społeczność programistów, co skutkuje bardzo szeroką ofertą różnych bibliotek i frameworków.
Dlatego kluczowym elementem narzędzi jak Kotlin/JS jest wsparcie dla tych aspektów języka.
Po pierwsze, Kotlin/JS pozwala na bezpośrednie użycie kodu JavaScript z Kotlina, poprzez użycie funkcji ,,js()'' \cite{kotlinlangKotlinDocs}.
Z jej użyciem programista może przekazać dowolny kod JavaScript, który zostanie wykonany w miejscu użycia funkcji.
Co więcej, Kotlin Multiplatform oferuje możliwość użycia bibliotek z menadżera pakietów NPM.
Wymaga to od programisty deklaracji tej zależności w konfiguracji Gradle, która jest podobna do deklaracji zwykłych zależności.
Po tym, może on użyć specjalnej annotacji ,,JsModule'' wraz z słowem kluczowym ,,external'', które pełnią rolę adaptera \cite{kotlinlangKotlinDocs}.
Wynika to z dynamicznego typowania JavaScript oraz stytycznego Kotlina. 
Dla zapewnienia lepszych doświadczeń programisty, Kotlin/JS umożliwia także wygenerowanie typów TypeScript \cite{kotlinlangKotlinDocs}.
Mogą być one użyte w przypadu np. udostępnienia biblioteki napisanej z użyciem Kotlin/JS.
Samo zapewnienie integracji z zewnętrznymi bibliotekami jest kluczowe w przypadku AWS Lambda, które wymaga integracji z np. AWS SDK.

Kolejnym ważnym aspektem AWS Lambda jest zmniejszanie wielkości artefaktu \cite{8116416}\cite{9095731}.
W przypadku Kotlin/JS oferowane jest kilka mechanizmów, które mogą pozytywnie wpłynąć na jego rozmiar.
Między innymi zapewnia on narzędzie DCE (ang. dead code elimination), które jest wbudowane w kompilator IR.
Polega ono na eliminacji kodu, który nie jest wykorzystywany w aplikacji.
Mogą to być na przykład funkcje z biblioteki standardowej Kotlina, które nie zostały użyte w kodzie funkcji.
Dodatkowo, aby kod Kotlin został przekonwertowany do kodu JavaScript, musi zostać użyta annotacja ,,@JsExport'' \cite{kotlinlangKotlinDocs}.
Dzięki temu, oznaczone nią funkcje czy klasy są traktowane jako elementy źródłowe dla mechanizmu DCE, co rozpocznie analizę używanego kodu właśnie od nich.
Dodatkowo, kompilator dokonuje minifikacji (ang. minification) nazw, jak zmienne.
Wszystko to pozwala na osiągnięcie mniejszego rozmiaru artefaktu, co jest istotnym aspektem w optymalizacji wydajności AWS Lambda.

Kotlin/JS posiada jednak także ograniczenia, które mogą negatywnie wpłynąć na rozwój oprogramowania.
Po pierwsze, mimo integracji z bibliotekami zewnętrznymi, może wymagać ona znacznych nakładów pracy.
Wynika to z użycia słowa kluczowego ,,external'' i potrzeby utrzymywania kodu, który zawiera typy bibliotek zewnętrznych.
Może to znacząco spowolnić wszelkie zmiany wersji bibliotek, które mogą zmienić swoje interfejsy.
Dodatkowo, Kotlin/JS posiada znaczne ograniczenia w mechanizmie refleksji.
Nie implementuje on całego API reflekcji Kotlina, a jednymie referencję klasy (::class), typy KType i KClass oraz powiązane z nimi funkcje ,,typeof()'' (zwracającą typ) i ,,createInstance()'' (tworzącą nową instancję klasy).
Czynniki te powinny być uwzględnione w przypadku wyboru Kotlin/JS jako technologii systemu działającego w AWS Lambda.

% 0. Wstęp

% 1. Sposób działania:
% Kotlin\JS jako docelowa platforma. Kod kotlin -> JavaScript
% Jak to działa? - Kompilator IR
% Wsparcie dla browser lub node.js

% 2. Integracja z JS:
% Wsparcie dla bibliotek i kodu JS z Kotlina
% Generowanie typów typescript

% 3. Wielkość artefaktu (ważne w AWS Lambda):
% Wsparcie dla webpacka (czy dla node.js też?)
% Dead Code Elimination i @JsExport
% Minification - jak to przetłumaczyć?

% 4. Ograniczenia:
% Reflection jako ograniczenie

% Linki:
% https://kotlinlang.org/docs/js-overview.html
% https://kotlinlang.org/docs/js-ir-compiler.html
% https://kotlinlang.org/docs/javascript-dce.html
% https://kotlinlang.org/docs/js-ir-compiler.html#minification-of-member-names-in-production
% https://kotlinlang.org/docs/js-ir-compiler.html#current-limitations-of-the-ir-compiler - końcówka, zawrzeć odnośnie JsExport
% https://kotlinlang.org/docs/js-ir-compiler.html#preview-generation-of-typescript-declaration-files-d-ts - typescript
% https://kotlinlang.org/docs/dynamic-type.html - do używania kodu JS z Kotlina
% https://kotlinlang.org/docs/using-packages-from-npm.html - używanie NPM
% https://kotlinlang.org/docs/js-to-kotlin-interop.html#kotlin-types-in-javascript - używanie kotlina z JS (?)
% https://kotlinlang.org/docs/js-reflection.html - reflection jako ograniczenie
\section{Kotlin/Native}\label{chapter:kotlin_native}

