\chapter{Przegląd literatury}\label{chapter:przeglad_literatury}

\section{Wyniki przeglądu}\label{chapter:przeglad_literatury_wyniki}

W ramach przeglądu wybrano X prac badawczych. Na bazie prac rozpatrzono postawione pytania badawcze. Odpowiedzi na nie zostały zawarte w kolejnych podrozdziałach.

\subsection{Czynniki wpływające na wydajność funkcji}\label{chapter:przeglad_literatury_wyniki_czynniki}

Pierwszym pytaniem badawczym postawionym do przeglądu literatury jest: ,,Jakie są główne czynniki wpływające na wydajność funkcji AWS Lambda?''.
Identyfikacja czynników wpływających na wydajność jest kluczowa w kontekście jej optymalizacji.
Pozwoli to następnie na zrozumienie na które z czynników ma także wpływ twórca funkcji AWS Lambda, co ułatwi dalszą analizę metod poprawy ich wydajności.

\subsubsection*{Wielkosć pamięci funkcji}

Kelly et al. \cite{9284261} zauważa, że wielkość pamięci funkcji oprócz bezpośredniego wpływu na czas działania funkcji, wywiera także wpływ na inne czynniki jak użycie procesora czy wydajność I/O dysku. 
W ramach badania wykonano pomiary dla funkcji bezserwerowych oferowanych przez wielu dostawców chmurowych, w tym Amazon Web Services, w celu zrozumienia infrastruktury i jej zarządzania, co domyślnie jest ukryte dla użytkownika funkcji jak AWS Lambda.   
Poprzez analizę maszyn wirtualnych, w ramach których uruchamiany jest kod funkcji, możliwe było otrzymanie wartości parametrów, które nie są domyślnie konfigurowalne podczas wdrażania funkcji przez programistę.

Pomiary pokrywały wiele parametrów funkcji, m. in. łączny czas wykonania, czas inicjalizacji, użycie procesora, wydajność I/O dysku oraz liczbę utworzonych maszyn wirtualnych (co wpływa na częstość zimnych startów). 
W badaniu uwzględniono predefiniowane wielkości pamięci, które mogą być wybrane przez programistę (128MB, 256MB, 512MB, 1024MB oraz 2048MB).
Wraz z wzrostem pamięci funkcji, parametry te poprawiały się.

Autorzy podkreślili ważność odpowiedniego doboru wielkości pamięci podczas tworzenia funkcji. Dodatkowo, wykazali, że platforma AWS nie używa ponownie maszyn wirtualnych w celu wykonania kolejnych zapytań funkcji, co powoduje częstsze zimne starty. Wykazano zatem, że te dwa czynniki (wielkość pamięci i zimne starty) znacząco wpływają na ogólną wydajność funkcji AWS Lambda.