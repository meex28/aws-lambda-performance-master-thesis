\section{Cel przeglądu}\label{chapter:cel_przegladu}

Celem przeglądu jest poznanie obecnego stanu wiedzy oraz wykorzystanie jej jako wsparcia w odpowiedzi na postawione w pracy pytania badawcza. 
Zamierzeniem przeglądu jest analiza trzech głównych obszarów:

\begin{enumerate}
    \item Czynników wpływających na wydajność AWS Lambda.
    \item Istniejących metod optymalizacji wydajności AWS Lambda dla ekosystemu Java.
    \item Charakterystyki rozwoju funkcji AWS Lambda z perspektywy pracy programisty.
\end{enumerate}

Obszary te zostały wybrane na podstawie pytań badawczych. 
Ich zrozumienie pozwoli na identyfikację zagadnień, które nie zostały wystarczająco zbadane, 
a mogą zawierać potencjalne metody usprawnienia wydajności funkcji AWS Lambda w ramach ekosystemu Java.
Dodatkowo, pozwoli to na przygotowanie badań, które będą lepiej odzwierciedlać rzeczywistą praktykę aplikacji bezserwerowych.

\section{Metodyka przeglądu literatury}\label{chapter:metodyka_przegladu}

Do wykonania przeglądu literatury wybrano metodykę szybkiego przeglądu (ang. rapid review). 
Szybkie przeglądy to metoda badań wtórnych stosowana w inżynierii oprogramowania, której celem jest szybka synteza wyników badań.
Ich głównym zadaniem jest dostarczanie aktualnych informacji opartych na dowodach.
Pomaga to praktykom podejmować decyzje dotyczące specyficznych problemów w ramach ich kontekstu pracy i ograniczeń czasowych.
Metoda ta często wykonywana jest przez jedną osobę, przy jednoczesnym użyciu bardziej restrykcyjnych kryteriów.
Jest to jednak świadoma strategia mająca na celu redukcję czasu i wysiłku \cite{cartaxo2020rapidreviewssoftwareengineering}.
Proces szybkiego przeglądu składa się z trzech etapów, które zostały zrealizowane w pracy:

\begin{enumerate}
    \item Zaplanowanie (określenie potrzeby przeglądu, problemu i pytań badawczych).
    \item Przeprowadzenie (stworzenie i wykonanie strategi wyszukiwania, procedur selekcji, oceny jakości, ekstrakcji i syntezy).
    \item Raportowanie (omówienie wyników przeglądu i odpowiedź na pytania badawcze).
\end{enumerate}

\section{Proces przeglądu}\label{chapter:proces_przegladu}

\subsection{Omówienie pytań badawczych}\label{chapter:omowanie_pytan_badawczych}

Formułowanie pytań badawczych to istotna część przeglądu \cite{KitchenhamProceduresSR}. 
Pytania te, oparte na celu pracy, wskazują kierunek przy opracowywaniu i wdrażaniu kryteriów przeglądu. 
W ramach wykonanego przeglądu literatury postawiono konkretne pytania, mające na celu ukierunkowanie analizy, 
co umożliwi odpowiedź na główne pytania badawcze całej pracy. 
Do każdego z nich dołączono krótkie wyjaśnienie i motywację.
 
\begin{itemize}
    \item PB1: Jakie są główne czynniki wpływające na wydajność funkcji AWS Lambda?
    
    Odpowiedź na postawione pytanie ma na celu zrozumienie sposobu działania AWS Lambda pod kątem wydajności.
    Powyższe pytanie nie odnosi się wyłącznie do technologii z środowiska Java, co pozwoli na poszerzenie analizy.
    Dostarczone odpowiedzi będą wsparciem dla znalezienia nowych metod optymalizacji wydajności, ze względu na zrozumienie na jakie czynniki owe metody mogą wpływać.
    Dodatkowo, zidentyfikowane czynniki pozwolą na przygotowanie bardziej jakościowych badań. 
    Badania będą mogły być realizowane w ramach różnych scenariuszy, które będą wykonane dla różnych wartości znalezionych parametrów.  

    \item PB2: Jakie są istniejące metody optymalizacji wydajności funkcji AWS Lambda działających w ekosystemie Java?
    
    Pytanie pozwoli ustalić jaki jest istniejący stan wiedzy dla metod optymalizacji wydajności funkcji AWS Lambda w ramach środowiska Java.
    W ramach tego pytania analiza skupi się na szeroko pojętym ekosystemie Java, czyli różnych językach wywodzących się z Javy, bibliotekach, frameworkach i środowiskach wykonawczych.
    Analiza pozwoli na wskazanie obszarów, które nie zostały jeszcze zbadane lub zostały zbadane niewystarczająco.
    Tak wyznaczone zagadnienia będą potencjalnym miejscem poszukiwania nowych metod.
    Stanowi to znaczącą pomoc w realizacji celu pracy.

    \item PB3: Jakie są cechy rozwoju aplikacji w architekturze bezserwerowej AWS Lambda?
    
    Ostatnie pytanie skupia się na perspektywie programisty tworzącego aplikacje opierające się o funkcje bezserwerowe oferowane przez AWS.
    Zdecydowano się na przegląd literatury w tym zakresie ze względu na specyficzne podejście do tworzenia takich aplikacji.
    Rozpatrzenie tego pozwoli następnie na analizę zaproponowanych metod optymalizacji pod względem ich wpływu na proces wytwarzania oprogramowania.
    Przewiduje się, że analiza ta może być skomplikowana z powodu zróżnicowanych praktyk budowy wspomnianych systemów.

\end{itemize}

\subsection{Przeszukiwane zasoby}\label{chapter:przeszukiwane_zasoby}

\subsection{Wyszukiwane terminy}\label{chapter:wyszukiwane_terminy}

\subsection{Selekcja literatury}\label{chapter:selekcja_literatury}

\subsection{Ocena jakości}\label{chapter:ocena_jakosci}
