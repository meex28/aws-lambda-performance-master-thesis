\chapter*{Wstęp}\label{chapter:wstep}
\addcontentsline{toc}{chapter}{Wstęp}  

TODO: Do napisania na końcu

\section*{Problem badawczy}\label{chapter:problem_badawczy}

Usługa AWS Lambda jest jednym z kluczowych serwisów oferowanych przez chmurę Amazon Web Services (AWS) w architekturze bezserwerowej. Wraz z rosnącą popularnością tego rodzaju architektur, pojawia się potrzeba ciągłej poprawy ich działania. Jednym z pierwotnie dostępnych języków programowania w AWS Lambda jest Java (oraz inne języki oparte o Java Virtual Machine), która jednak ze względu na swoją specyfikę (w tym wpływ na responsywność funkcji) nie jest najczęściej wybieraną opcją implementacji w tym serwisie.

Ważnym elementem pracy z AWS Lambda jest wydajność tworzonych funkcji, która przekłada się bezpośrednio na koszt usługi. Wpływ na wydajność mają takie czynniki jak na przykład tzw. zimne i ciepłe starty, czy czas działania samej funkcji, a niewystarczająco szybkie mogą powodować trudności dla programistów.

Mimo rozwoju wielu różnych technologii, ekosystem Java dalej cieszy się dużą popularnością wśród zespołów programistycznych. Poprawa wydajności funkcji AWS Lambda w tym ekosystemie ułatwi programistom decyzję o wyborze tego rozwiązania oraz pozwoli na lepszą pracę z już znanym językiem. Z tego powodu istnieje potrzeba analizy i zaproponowania metod optymalizacji wydajności dla funkcji AWS Lambda w ekosystemie Java.

\section*{Cel pracy}\label{chapter:cel_pracy}

Celem pracy jest zaproponowanie nowych metod poprawy wydajności funkcji AWS Lambda w ekosystemie Java oraz analiza ich wpływu na czas działania funkcji i inne wybrane czynniki, które mogą wpłynąć na jakość pracy programistów. 

\section*{Pytania badawcze}\label{chapter:pytania_badawcze}

\begin{itemize}
    \item PB1: Które metody optymalizacji pozwalają na najlepszą poprawę czasu wykonania funkcji AWS Lambda w ekosystemie Java?
    \item PB2: W jakim stopniu wybrane metody optymalizacji redukują czas zimnego startu funkcji Java w AWS Lambda?
    \item PB3: Jakie kompromisy w procesie rozwoju oprogramowania wiążą się implementacją poszczególnych metod optymalizacji wydajności funkcji Java w AWS Lambda?
\end{itemize}

\section*{Zakres pracy}\label{chapter:zakres_pracy}

Cel pracy zostanie zrealizowany poprzez następujące działania stanowiące zasadniczy wkład pracy:
\begin{enumerate}
    \item Systematyczny przegląd literatury
    \item Identyfikacja i zaproponowanie metod
    \item Analiza wpływu
\end{enumerate}

\section*{Struktura pracy}\label{chapter:struktura_pracy}
TODO: Do napisania na końcu