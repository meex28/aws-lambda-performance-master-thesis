\chapter*{Wstęp}\label{chapter:wstep}
\addcontentsline{toc}{chapter}{Wstęp}  

Do napisania na końcu

\section*{Problem badawczy}\label{chapter:problem_badawczy}

Usługa AWS Lambda jest jednym z kluczowych serwisów oferowanych przez chmurę Amazon Web Services (AWS) w architekturze bezserwerowej. Wraz z rosnącą popularnością tego rodzaju architektur, pojawia się potrzeba ciągłej poprawy ich działania. Jednym z pierwotnie dostępnych języków programowania w AWS Lambda jest Java (oraz inne języki oparte o Java Virtual Machine), która jednak ze względu na swoją specyfikę (w tym wpływ na responsywność funkcji) nie jest najczęściej wybieraną opcją implementacji w tym serwisie.

Ważnym elementem pracy z AWS Lambda jest wydajność tworzonych funkcji, która przekłada się bezpośrednio na koszt usługi. Wpływ na wydajność mają takie czynniki jak na przykład tzw. zimne i ciepłe starty, czy czas działania samej funkcji, a niewystarczająco szybkie mogą powodować trudności dla programistów.

Mimo rozwoju wielu różnych technologii, ekosystem Java dalej cieszy się dużą popularnością wśród zespołów programistycznych. Poprawa wydajności funkcji AWS Lambda w tym ekosystemie ułatwi programistom decyzję o wyborze tego rozwiązania oraz pozwoli na lepszą pracę z już znanym językiem. Z tego powodu istnieje potrzeba analizy i zaproponowania metod optymalizacji wydajności dla funkcji AWS Lambda w ekosystemie Java.

\section*{Cel pracy}\label{chapter:cel_pracy}

Celem pracy jest zaproponowanie nowych metod poprawy wydajności funkcji AWS Lambda w ekosystemie Java oraz analiza ich wpływu na czas działania funkcji i inne wybrane czynniki, które mogą wpłynąć na jakość pracy programistów. 

\section*{Pytania badawcze}\label{chapter:pytania_badawcze}

\begin{itemize}
    \item PB1: Które metody optymalizacji pozwalają na najlepszą poprawę czasu wykonania funkcji AWS Lambda w ekosystemie Java?
    \item PB2: W jakim stopniu wybrane metody optymalizacji redukują czas zimnego startu funkcji Java w AWS Lambda?
    \item PB3: Jakie kompromisy w procesie rozwoju oprogramowania wiążą się implementacją poszczególnych metod optymalizacji wydajności funkcji Java w AWS Lambda?
\end{itemize}

\section*{Zakres pracy}\label{chapter:zakres_pracy}

Cel pracy zostanie zrealizowany poprzez następujące działania stanowiące zasadniczy wkład pracy:
\begin{enumerate}
    \item Wykonanie systematycznego przeglądu literatury w konkteście wydajności i tworzenia funkcji AWS Lambda. 
    Przegląd ten pozwoli na zrozumienie aktualnego stanu wiedzy oraz rozpoznanie obszarów, który nie zostały jeszcze wystarczająco zbadane. 
    \item Identyfikacja i zaproponowanie metod optymalizacji wydajności funkcji AWS Lambda w ekosystemie Java. 
    Propozycja nowych metod będzie znacząco opierać się na wykonanym przeglądzie literatury. 
    Jest to kluczowy element pracy, który następnie pozwoli na analizę tych metod w celu odpowiedzenia na postawione pytania badawcze.
    \item Przygotowanie i wykonanie badania poprzez zmierzenie wpływu wybranych metod na wydajność funkcji.
    Etap ten będzie opierał się o metody nowo zaproponowane oraz wybrane z przeglądu literatury.
    Zbadanie obu tych rodzajów pozwoli na bardziej rzetelną ocenę nowo wybranych sposobów optymalizacji.
    Przygotowane badanie zostanie dokładnie opisane, co pozwoli na jego odtworzenie.
    \item Analiza wyników badania i wyciągnięcie wniosków, co pozwoli na ocenę wybranych metod optymalizacji. 
\end{enumerate}

\section*{Struktura pracy}\label{chapter:struktura_pracy}

Praca składa się ze wstępu, słownika pojęć i akronimów, pięciu rozdziałów, zakończenia oraz bibliografii.
W ramach wstępu zdefiniowano problem badawczy, cel pracy, pytania badawcze oraz zakres pracy.
W pierwszym rozdziale opisano teoretyczne podstawy działania AWS Lambda, a także znaczenie modeli bezserwerowych i funkcji jako usług (na których opiera się AWS Lambda).
W rozdziale drugim dokonano systematycznego przeglądu literatury, na bazie którego opisano aktualny stan wiedzy.
Odpowiedziano na pytania badawcze do przeglądu oraz zlokalizowano luki w literaturze, co zostało wykorzystane w następnym rozdziale.
W trzecim rozdziale zawarto opis wybranych metod optymalizacji, w tym nowo zaproponowanych.
W rozdziale czwartym dokładnie opisano przygotowanie badania oraz jego przebieg, co pozwoli na jego odtworzenie.
W piątym rozdziale przedstawiono wyniki wykonanego badania oraz dokonano ich analizy.
W rozdziale szóstym wyciągnięto wnioski na bazie wyników badania oraz przeprowadzono ich dyskusję.
Zakończenie zostało wykonane jako podsumowanie całej pracy.
Pracę zakończono bibliografią, w tym wykazem cytowanej literatury oraz spisami rysunków, tabel i kodów źródłówych.
