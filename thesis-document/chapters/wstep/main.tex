\chapter*{Wstęp}\label{chapter:wstep}
\addcontentsline{toc}{chapter}{Wstęp}  

Usługa AWS Lambda jest jednym z kluczowych serwisów oferowanych przez chmurę Amazon Web Services (AWS) w architekturze bezserwerowej. 
Wraz z rosnącą popularnością tego rodzaju architektur, pojawia się potrzeba ciągłej poprawy ich działania. 
Jednym z pierwotnie dostępnych języków programowania w AWS Lambda jest Java oraz inne języki oparte o jej maszynę wirtualną.
Według raportu Datadog ,,The State of Serverless'' około 10\% wywołań funkcji AWS Lambda jest obsługiwane przez język Java \cite{datadog_state_of_serverless}.
Języki oparte o maszynę wirtualną Java są jednak wykorzystywane o wiele częściej.
Według ankiety StackOverflow w 2024 roku, około 30\% profesjonalnych programistów zadeklarowało, że korzysta z Javy.
Dodatkowo prawie 9.9\% z nich oznajmiło korzystanie z Kotlina, 3.8\% z Groovy, a 2.9\% z języka Scala. 

Wydajność jest jednym z najbardziej istotnych aspektów usług bezserwerowych.
Wynika to z specyficznego modelu tych usług, gdzie opłacany jest wyłącznie czas realnego korzystania z infrastruktury (tzw. ,,płać za użycie'', ang. ,,pay-per-use'') \cite{ServerlessApplicationsWhyWhenAndHow}.
Ze względu na jego wyjątkowe cechy i bezpośredni wpływ na koszt działania, opis tego modelu zostanie także zawarty w pracy.
Istotę omawianego aspektu podkreśla także fakt, że znaczącym czynnikiem w wyborze usług bezserwerowych jest właśnie redukcja kosztów \cite{ServerlessApplicationsWhyWhenAndHow}, która jest bezpośrednio powiązana z efektywnością.
Wspomniany raport Datadog ,,The State of Serverless'' \cite{datadog_state_of_serverless} dwukrotnie porusza problemy wydajnościowe w konkteście ekosystemu Java.
Po pierwsze podkreśla znacznie dłuższe zimne starty w porównaniu z innymi językami programowania, a także alokację większych rozmiarów pamięci dla funkcji AWS Lambda opartych o Javę.
Z tego powodu redukcja wpływu tych problemów może znacznie ułatwić korzystanie z AWS Lambda w ekosystemie Java, a co za tym idzie zwiększyć popularność tego języka programowania jako narzędzia dla usług bezserwerowych.

Jednymi z metod optymalizacji wydajności Javy w ramach AWS Lambda mogą być usługa SnapStart oraz kompilacja GraalVM.
W ramach przeglądu literatury odnaleziono jedynie pojedyncze prace opisujące ich użycie w kontekście AWS Lambda \cite{menéndez2023performancebestpracticesusing}\cite{ritzal2020optimizing}.
Dodatkowo poruszają one efektywność tych metod jedynie poprzez proste metryki jak czas wykonania ciepłych i zimnych startów, a także nie badają ich skuteczności pod względem rozmiaru pamięci funkcji.
Co więcej, wśród znalezionych prac proponujących metody optymalizacji wydajności dla AWS Lambda \cite{8116416}\cite{10.1145/3377812.3382135}\cite{9860368}\cite{FerreiraDosSantos2023}\cite{10.1145/3458336.3465305}\cite{menéndez2023performancebestpracticesusing}\cite{ritzal2020optimizing},
jedynie pojedyncza praca poruszała temat wpływu na jakość rozwoju oprogramowania \cite{10.1145/3377812.3382135}.
Jest to jednak istotny element, który powinien być wzięty pod uwagę podzas wyboru odpowiedniego sposobu poprawy efektywności funkcji.

Nie odnaleziono także prac badających wydajność użycia w AWS Lambda, alternatywnych języków programowania opartych o maszynę wirtualną Javy.
Istotną motywacją do badań w tym zakresie jest popularność języka Kotlin, którego użycie w ankiecie StackOverflow \cite{stackoverflow_survey_2024} zadeklarowało 9.9\% profesjonalnych programistów.
Sam Kotlin zapewnia szeroki wybór narzędzi oraz bibliotek. W ramach tego języka rozwijany jest także projekt Kotlin Multiplatform, który może zostać potencjalnie wykorzystany dla AWS Lambda.
Zatem język ten stanowi szeroki obszar do badań w kontekście wydajności w usłudze AWS Lambda.

\section*{Problem badawczy}\label{chapter:problem_badawczy}

Usługa AWS Lambda oferuje możliwość implementacji funkcji w językach z ekosystemu Java.
Zespoły programistyczne korzystające z tych rozwiązań mogą jednak doświadczać problemów wydajnościowych, które mogą być kluczowe w modelu bezserwerowym.
Niska wydajność funkcji może mieć wpływ na koszt infrastruktury oraz zadowolenie użytkowników systemu, którzy oczekują niskich czasów odpowiedzi na zapytania.
Istnieje potrzeba analizy i zaproponowania metod, które pozwolą zespołom programistycznym na poprawę wydajności tworzonych funkcji.

Przypadki użycia systemów opartych o funkcje AWS Lambda są bardzo róznorodne.
Z tego powodu twórcy takich systemów potrzebują odpowiednich informacji na temat metod, które potencjalnie zastosują.
Rodzi to zapotrzebowanie na badanie metod pod kątem różnych typów funkcji, a także ich potencjalnego wpływu na proces rozwoju oprogramowania.
Zastosowanie nieodpowiedniej metody może prowadzić do niespodziewanych ograniczeń w rozwoju i utrzymaniu funkcji AWS Lambda.

\section*{Cel pracy}\label{chapter:cel_pracy}

Celem pracy jest zaproponowanie nowych metod poprawy wydajności funkcji AWS Lambda w ekosystemie Java oraz analiza ich wpływu na wydajność funkcji i inne wybrane czynniki, które mogą wpłynąć na jakość pracy programistów. 

\section*{Pytania badawcze}\label{chapter:pytania_badawcze}

\begin{itemize}
    \item PB1: Które metody optymalizacji pozwalają na najlepszą poprawę ogólnej wydajności funkcji AWS Lambda w ekosystemie Java?
    \item PB2: W jakim stopniu wybrane metody optymalizacji redukują czas zimnego startu funkcji Java w AWS Lambda?
    \item PB3: Jakie kompromisy w procesie rozwoju oprogramowania wiążą się implementacją poszczególnych metod optymalizacji wydajności funkcji Java w AWS Lambda?
\end{itemize}

\section*{Zakres pracy}\label{chapter:zakres_pracy}

Cel pracy zostanie zrealizowany poprzez następujące działania stanowiące zasadniczy wkład pracy:
\begin{enumerate}
    \item Wykonanie systematycznego przeglądu literatury w konkteście wydajności i tworzenia funkcji AWS Lambda. 
    Przegląd ten pozwoli na zrozumienie aktualnego stanu wiedzy oraz rozpoznanie obszarów, który nie zostały jeszcze wystarczająco zbadane. 
    \item Identyfikacja i zaproponowanie metod optymalizacji wydajności funkcji AWS Lambda w ekosystemie Java. 
    Propozycja nowych metod będzie znacząco opierać się na wykonanym przeglądzie literatury. 
    Jest to kluczowy element pracy, który następnie pozwoli na analizę tych metod w celu odpowiedzenia na postawione pytania badawcze.
    \item Przygotowanie i wykonanie badania poprzez pomiar wydajności funkcji korzystających z wybranych metod.
    Etap ten będzie opierał się o nowo zaproponowane metody oraz te wybrane z przeglądu literatury.
    Zbadanie obu rodzajów sposobów pozwoli na bardziej rzetelną ocenę nowo wybranych metod optymalizacji.
    Przygotowane badanie zostanie dokładnie opisane, a jego kod źródłowy udostępniony w formie repozytorium, co pozwoli na jego odtworzenie.
    \item Analiza wyników badania i wyciągnięcie wniosków, co pozwoli odpowiedzieć na sformułowane w pracy pytania badawcze. 
\end{enumerate}

\section*{Struktura pracy}\label{chapter:struktura_pracy}

Praca składa się ze wstępu, słownika pojęć i akronimów, sześciu rozdziałów, zakończenia oraz bibliografii.
W ramach wstępu zdefiniowano problem badawczy, cel pracy, pytania badawcze oraz zakres pracy.
W pierwszym rozdziale opisano teoretyczne podstawy działania AWS Lambda, a także znaczenie modeli bezserwerowych i funkcji jako usług (o które opiera się AWS Lambda).
W rozdziale drugim dokonano systematycznego przeglądu literatury, na bazie którego opisano aktualny stan wiedzy.
Odpowiedziano na pytania badawcze do przeglądu oraz zlokalizowano luki w literaturze, co zostało wykorzystane w następnym rozdziale.
W trzecim rozdziale zawarto opis wybranych metod optymalizacji, w tym nowo zaproponowanych.
W rozdziale czwartym dokładnie opisano przygotowanie badania oraz jego przebieg, co pozwoli na jego odtworzenie.
W piątym rozdziale przedstawiono wyniki wykonanego badania oraz dokonano obiektywnej analizy poszczególnych metryk.
W rozdziale szóstym dokonano szerszej analizy wyników badania, wyciągnięto wnioski oraz przeprowadzono ich dyskusję w celu odpowiedzi na sformułowane pytania badawcze.
Zakończenie zostało wykonane jako podsumowanie całej pracy oraz wyznaczenie dalszych perspektyw rozwoju.
Pracę zakończono bibliografią, w tym wykazem cytowanej literatury oraz spisami rysunków i tabel.
